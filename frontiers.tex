%%%%%%%%%%%%%%%%%%%%%%%%%%%%%%%%%%%%%%%%%%%%%%%%%%%%%%%%%%%%%%%%%%%%%%%%%%%%%%%%%%%%%%%%%%%%%%%%%%%%%%%%%%%%%%%%%%%%%%%%%%%%%%%%%%%%%%%%%%%%%%%%%%%%%%%%%%%
% This is just an example/guide for you to refer to when submitting manuscripts to Frontiers, it is not mandatory to use Frontiers .cls files nor frontiers.tex  %
% This will only generate the Manuscript, the final article will be typeset by Frontiers after acceptance.   
%                                              %
%                                                                                                                                                         %
% When submitting your files, remember to upload this *tex file, the pdf generated with it, the *bib file (if bibliography is not within the *tex) and all the figures.
%%%%%%%%%%%%%%%%%%%%%%%%%%%%%%%%%%%%%%%%%%%%%%%%%%%%%%%%%%%%%%%%%%%%%%%%%%%%%%%%%%%%%%%%%%%%%%%%%%%%%%%%%%%%%%%%%%%%%%%%%%%%%%%%%%%%%%%%%%%%%%%%%%%%%%%%%%%

%%% Version 3.4 Generated 2018/06/15 %%%
%%% You will need to have the following packages installed: datetime, fmtcount, etoolbox, fcprefix, which are normally inlcuded in WinEdt. %%%
%%% In http://www.ctan.org/ you can find the packages and how to install them, if necessary. %%%
%%%  NB logo1.jpg is required in the path in order to correctly compile front page header %%%

\documentclass[utf8]{frontiersSCNS} % for Science, Engineering and Humanities and Social Sciences articles
%\documentclass[utf8]{frontiersHLTH} % for Health articles
%\documentclass[utf8]{frontiersFPHY} % for Physics and Applied Mathematics and Statistics articles

%\setcitestyle{square} % for Physics and Applied Mathematics and Statistics articles
\usepackage{url,hyperref,lineno,microtype,subcaption}
\usepackage[onehalfspacing]{setspace}

\linenumbers

% Added by RJCB
% Subfigures
%\usepackage{subcaption} % Already added

% Added by RJCB
% Subfigure numbers use uppercase, 
% from https://tex.stackexchange.com/a/278140
\renewcommand{\thesubfigure}{\Alph{subfigure}}

% Added by RJCB
\usepackage{listings}
\usepackage{hyperref}
\usepackage{todonotes}
\usepackage{verbatim}
\usepackage{pgf}
\usepackage{bm}
\usepackage{multirow}
\usepackage{amsfonts}
\usepackage{array}
\usepackage{booktabs}
\newcolumntype{C}[1]{>{\centering\arraybackslash}p{#1}}
\newcolumntype{L}[1]{>{\raggedright\arraybackslash}p{#1}}
\usepackage{longtable}

% Added by RJCB
%\usepackage{xr}
%\externaldocument{frontiers_SupplementaryMaterial.tex}

% Leave a blank line between paragraphs instead of using \\


\def\keyFont{\fontsize{8}{11}\helveticabold }
\def\firstAuthorLast{Sample {et~al.}} %use et al only if is more than 1 author
\def\Authors{Richèl J.C. Bilderbeek\,$^{1}$, Maksim V. Baranov\,$^{1}$, Geert van den Bogaart\,$^{1}$ and Frans Bianchi\,$^{1,*}$}


% Affiliations should be keyed to the author's name with superscript numbers and be listed as follows: Laboratory, Institute, Department, Organization, City, State abbreviation (USA, Canada, Australia), and Country (without detailed address information such as city zip codes or street names).
% If one of the authors has a change of address, list the new address below the correspondence details using a superscript symbol and use the same symbol to indicate the author in the author list.
\def\Address{$^{1}$Molecular Immunology, GBB, University of Groningen, Groningen, The Netherlands}
% Provide the exact contact address (this time including street name and city zip code) and email of the corresponding author
\def\corrAuthor{Frans Bianchi, Nijenborgh 7, 9747 AG Groningen}

\def\corrEmail{f.bianchi@rug.nl}




\begin{document}
\onecolumn
\firstpage{1}

\title[Transmembrane helices are an over-presented and evolutionarily conserved source of major histocompatibility complex class I and II epitopes
]{Transmembrane helices are an over-presented and evolutionarily conserved source of major histocompatibility complex class I and II epitopes
} 

\author[\firstAuthorLast ]{\Authors} %This field will be automatically populated
\address{} %This field will be automatically populated
\correspondance{} %This field will be automatically populated

\extraAuth{}% If there are more than 1 corresponding author, comment this line and uncomment the next one.
%\extraAuth{corresponding Author2 \\ Laboratory X2, Institute X2, Department X2, Organization X2, Street X2, City X2 , State XX2 (only USA, Canada and Australia), Zip Code2, X2 Country X2, email2@uni2.edu}


\maketitle


\begin{abstract}

%%% Leave the Abstract empty if your article does not require one, please see the Summary Table for full details.
\section{}
Cytolytic T cell responses are predicted to be biased towards membrane proteins. 
The peptide-binding grooves of most alleles
of histocompatibility complex class I (MHC-I) are relatively hydrophobic, 
therefore peptide fragments derived from human transmembrane helices (TMHs) are predicted to be presented more often as would be expected 
based on their abundance in the proteome. However, the physiological reason of why membrane proteins might be over-presented is unclear. 
In this study, we show that the predicted over-presentation of TMH-derived peptides is general, as it is predicted for bacteria and viruses 
and for both MHC-I and MHC-II, and confirmed by re-analysis of epitope databases. Moreover, we show that TMHs are evolutionarily more conserved, 
because single nucleotide polymorphisms (SNPs) are present 
relatively less frequently in TMH-coding chromosomal regions 
compared to regions coding for extracellular and cytoplasmic protein regions. 
Thus, our findings suggest that 
both cytolytic and helper T cells are more tuned to respond to membrane proteins, 
because these are evolutionary more conserved. 
We speculate that TMHs are less prone to mutations 
that enable pathogens to evade T cell responses.


\tiny
 \keyFont{ \section{Keywords:} antigen presentation, membrane proteins,adaptive immunity, transmembrane domain, epitopes, MHC-I, MHC-II, evolutionary conservation}
\end{abstract}

%%%%%%%%%%%%%%%%%%%%%%%%%%%%%%%%%%%%%%%%%%%%%%%%%%%%%%%%%%%%%%%%%%%%%%%%%%%%%%%%
\section{Introduction}
%%%%%%%%%%%%%%%%%%%%%%%%%%%%%%%%%%%%%%%%%%%%%%%%%%%%%%%%%%%%%%%%%%%%%%%%%%%%%%%%

% For Original Research Articles \citep{conference}, Clinical Trial Articles \citep{article}, and Technology Reports \citep{patent}, the introduction should be succinct, with no subheadings \citep{book}. For Case Reports the Introduction should include symptoms at presentation \citep{chapter}, physical exams and lab results \citep{dataset}.

% \paragraph{Immune response}

Our immune system fights diseases and infections from pathogens, 
such as fungi, bacteria or viruses. 
An important part of the acquired immune response, 
that develops specialized and more specific recognition of pathogens 
than the innate immune response, 
are T cells which recognize peptides, called epitopes, derived from antigenic proteins presented on Major Histocompatibility Complexes (MHC) class I and II on the cell surface. 

% \paragraph{Classification of HLA}

The MHC proteins are heterodimeric complexes encoded by the
HLA (Human Leukocyte Antigens) genes.
In humans, the peptide binding groove of MHC-I is made by only the alpha subunit. There are three classical alleles of MHC-I, hallmarked by a highly polymorphic alpha chain called HLA-A, HLA-B and HLA-C, that all present epitopes to cytolytic T cells. 
For MHC-II, both the alpha and the beta chains contribute to the peptide binding groove. There are three classical alleles of MHC-II as well, called HLA-DR, HLA-DQ and HLA-DP, that all present epitopes to helper T cells.
Each MHC complex can present a subset of all possible peptides.
For example, HLA-A and HLA-B have no overlap in which
epitopes they bind \citep{lund2004definition}.
Moreover, the HLA genes of humans are highly polymorphic, with hundreds 
to thousands of different alleles, 
and each different allele presents a different subset of peptides \citep{marsh2010nomenclature}.

% \paragraph{HLAs increase detection range}

Humans express a limited set of MHC alleles
and therefore an individual's immune system detects 
only a fraction of all possible peptide fragments. 
However, at the population level, the coverage of pathogenic peptides that are detected 
is very high, because of the highly polymorphic MHC genes.
It is therefore believed that MHC polymorphism improves immunity at the population level, 
as mutations in a protein that disrupt a particular MHC presentation at the individual level, 
so-called escape mutations, 
will not affect MHC presentation for all alleles present in the population \citep{sommer2005importance}.

% \paragraph{Epitope prediction}

Many studies are aimed at identifying the repertoire of epitopes that are presented 
in any of the different alleles to determine which epitopes will result in an immune response, 
as this will for instance aid the design of vaccines. 
These studies have led to the development of prediction algorithms 
that allow for very reliable \emph{in silico} 
predictions of the peptide binding affinities
\citep{larsen2010identification,schellens2008unanticipated,tang2011genome}.
For example, S. Tang et al. \citep{tang2011genome} found that, 
of the 432 peptides that were predicted to bind to an MHC allele,
86\% were experimentally confirmed to do so. 

% \paragraph{TMHs}

Using these prediction algorithms, 
we recently showed that peptides derived 
from transmembrane helices (TMHs) 
are likely to be more frequently presented by MHC-I 
than expected based on their abundance \citep{bianchi2017},
which is in line with a previous study 
by Istrail et al \citep{istrail2004comparative},
demonstrating that N-terminal signal sequences 
are likely to be presented within major histocompatibility complexes, 
due their hydrophobic nature. 
Moreover, we showed that some well-known immunodominant peptides stem from TMHs. 
This over-presentation is attributed to the fact 
that the peptide-binding groove of most MHC-I alleles
is relatively hydrophobic, 
and therefore hydrophobic TMH-derived peptides have a higher affinity 
to bind than their soluble hydrophilic 
counterparts. 

TMHs are hydrophobic 
as they need to span the hydrophobic lipid bilayer of cellular membranes.
They consist of an alpha helix of, on average, 23 amino acids in length. 
TMHs can also be predicted with high accuracy from a protein sequence 
by bioinformatics approaches \citep{krogh2001predicting,kall2004combined,arai2004conpred,jones2007improving,klammer2009metatm,wang2019efficient}. 
For example, a study by Jones \citep{jones2007improving} found that,
from 184 transmembrane proteins (TMPs) with known topology, 
80\% of the TMH predictions of these proteins
matched the experimental findings.
TMHs are common structures in the proteins of humans and microbes. 
Different TMH prediction tools estimate that 15-39\% of all proteins 
in the human proteome contain at least one TMH \citep{ahram2006estimation}.
However, the physiological reason why peptides derived from TMHs 
would be presented more often than peptides 
stemming from soluble (i.e., extracellular or cytoplasmic) protein regions is unknown. 
In this study, we hypothesized that the presentation of 
TMH residues is evolutionarily preferred, 
since TMHs are less prone to undergo escape mutations. 
One reason to expect such a reduced 
variability (and hence evolutionary conservation) in TMHs, 
is that these are restricted in their variability 
by the functional requirement to span a lipid bilayer. 
This limits 
many of the amino acids present in TMHs 
to have hydrophobic side chains 
\citep{hessa2007molecular,jones1994model}.
Therefore, we speculated that the TMHs of pathogens 
might have a lower chance to develop escape mutations, 
as that will result in a dysfunctional TMH 
and render the protein inactive.

This study had two objectives. 
First, we aimed to generalize our findings by predicting
the antigenic presentation
from different kingdoms of life in both MHC-I and -II. 
From these \emph{in silico} predictions, we conclude that TMH-derived
epitopes from a human, viral and bacterial proteome
are likely to be presented more often than expected by chance 
for most alleles of MHC-I and II. 
We confirmed the presentation of TMH-derived peptides 
by re-analysis of peptides from The Immune Epitope Database (IEDB) \citep{vita2019immune}.
Second, we tested our hypothesis that TMHs 
are more evolutionary conserved than soluble protein regions.
Our analysis of human single nucleotide polymorphisms (SNPs) showed 
that random point mutations are indeed less likely
to occur within TMHs. 
These findings strengthen the emerging notion 
that TMHs are important for the T cell-mediated adaptive immune system, 
and hence are of  
overlooked importance in vaccine development.

%%%%%%%%%%%%%%%%%%%%%%%%%%%%%%%%%%%%%%%%%%%%%%%%%%%%%%%%%%%%%%%%%%%%%%%%%%%%%%
\section{Methods}
%%%%%%%%%%%%%%%%%%%%%%%%%%%%%%%%%%%%%%%%%%%%%%%%%%%%%%%%%%%%%%%%%%%%%%%%%%%%%%

%%%%%%%%%%%%%%%%%%%%%%%%%%%%%%%%%%%%%%%%%%%%%%%%%%%%%%%%%%%%%%%%%%%%%%%%%%%%%%
\subsection{Predicting TMH epitopes}
%%%%%%%%%%%%%%%%%%%%%%%%%%%%%%%%%%%%%%%%%%%%%%%%%%%%%%%%%%%%%%%%%%%%%%%%%%%%%%

To predict how frequently epitopes overlapping with TMHs are presented,
a similar analysis strategy was applied as described in \citep{bianchi2017} 
for several alleles of both MHC-I and MHC-II, 
and for a human, viral and bacterial proteome.
To summarize, for each proteome, 
all possible 9-mers (for MHC-I) or 14-mers (MHC-II) were derived. 
For each of these peptides, we determined if it overlapped with a predicted 
TMH and if it was predicted to bind to the most frequent alleles of each MHC allele.

For MHC-I, 9-mers were used, 
as this is the length most frequently presented in MHC-I 
and was used in our earlier study \citep{bianchi2017}. 
For MHC-II, 14-mers were used, 
as this is the most frequently occurring epitope length \citep{bergseng2015different}.
A human (UniProt ID UP000005640\_9606), 
viral (SARS-CoV-2, UniProt ID UP000464024) 
and bacterial (\emph{Mycobacterium tuberculosis}, UniProt ID UP000001584) 
reference proteome was used. 
\verb;TMHMM; \citep{krogh2001predicting} was used to predict the topology 
of the proteins within these proteomes.
To predict the affinity of an epitope to a certain HLA allele,
 \verb;EpitopePrediction; \citep{bianchi2017} for MHC-I 
and \verb;MHCnuggets; \citep{shao2020high} for MHC-II was used.
Both MCH-I and MHC-II alleles were selected 
to have a high prevalence in the population,
where the alleles of MHC-I are the alleles representing the 13 supertypes 
with over 99.6\% coverage of the population's MHC-I repertoire as defined by \citep{lund2004definition} \citep{sette1999},
and the 21 MHC-II alleles, have a phenotypic frequency 
of 14\% or more in the human population \citep{greenbaum2011functional}.
 
We define a protein to be a binder if, for a certain MHC allele, 
any of its 9-mer or 14-mer peptides have an IC50 value in the lowest 2\% of 
all peptides within a \emph{proteome} 
(see supplementary Tables ST1 and ST2 for values), 
this differs from our previous study where we defined
a binder as having an IC50 in the lowest 2\% 
of the peptides within a \emph{protein}.
% See https://github.com/richelbilderbeek/bianchi_et_al_2017/blob/72e6755a31d400158368509fd80a41e984677ab1/predict-binders.R#L17
This revised definition precludes bias of proteins 
that give rise to no or only very few MHC epitopes.
To verify that the slight change in method yields similar results,
a side by side comparison is shown in the supplementary materials, 
Figures S1 and S2.

%%%%%%%%%%%%%%%%%%%%%%%%%%%%%%%%%%%%%%%%%%%%%%%%%%%%%%%%%%%%%%%%%%%%%%%%%%%%%%
\subsection{TMH epitopes obtained from experimental data}\label{subsec:elution_studies}
%%%%%%%%%%%%%%%%%%%%%%%%%%%%%%%%%%%%%%%%%%%%%%%%%%%%%%%%%%%%%%%%%%%%%%%%%%%%%%

To obtain experimental confirmation that peptides stemming from TMHs 
are presented by MHC-I and MHC-II,
we mined the IEDB \citep{vita2019immune}
for confirmed human MHC-ligands.
We queried the IEDB for all linear epitopes obtained
from MHC ligand assays in healthy humans, 
carrying the MHC alleles as used in this study.
From these epitopes, we kept those that were present
exactly once in the human reference proteome
with UniProt ID UP000005640\_9606.
We concluded that the epitope overlapped with a TMH if at least 1 amino acid was overlapping with a TMH, as predicted with \verb;TMHMM; \cite{krogh2001predicting}.

The full analysis can be found
at \url{https://github.com/richelbilderbeek/bbbq_article_issue_157}.

%%%%%%%%%%%%%%%%%%%%%%%%%%%%%%%%%%%%%%%%%%%%%%%%%%%%%%%%%%%%%%%%%%%%%%%%%%%%%%
\subsubsection{Evolutionary conservation of TMHs}
%%%%%%%%%%%%%%%%%%%%%%%%%%%%%%%%%%%%%%%%%%%%%%%%%%%%%%%%%%%%%%%%%%%%%%%%%%%%%%

% \paragraph{Introduction}

To determine the evolutionary conservation of TMHs, we first
collected human single nucleotide polymorphisms (SNPs)
resulting in a single amino acid substitution and determined if
this occurred within a predicted TMH or not.

% \paragraph{Data}

As a data source, multiple
NCBI (\url{https://www.ncbi.nlm.nih.gov/}) databases were used: 
the \emph{dbSNP} \citep{sherry2001dbsnp} database,
which contains 650 million 
cataloged non-redundant human variations (called \emph{RefSNPs},
\url{https://www.ncbi.nlm.nih.gov/snp/docs/RefSNP_about/}), and the databases \emph{gene} (for gene names \citep{brown2015gene})
and \emph{protein} (for proteins sequences \citep{sayers2010database}).

% \paragraph{Pipeline}

The first query was a call to the \emph{gene} database for the 
term 'membrane protein' (in all fields) 
for the organism \emph{Homo sapiens}.
This resulted in 1,077 gene IDs (on December 2020).
% 1077 gene IDs is correct for December 2020.
% At 2021-03-01, one will get 1130 gene IDs.
% Also, one of the gene IDs that was valid back then,
% has been obsoleted.
The next query was a call to the \emph{gene} database 
to obtain the gene names from the gene IDs.
Per gene name, the \emph{dbSNP} NCBI database was queried for 
variations associated with the gene name. 
As the NCBI API constrains its users to three calls per second
(to assure fair use), we had to limit the extent of our analysis.

The number of SNPs was limited to the first 250 variations per gene,
resulting in $\approx$61k variations.
Only variations that result in a SNP for
a single amino acid substitution were analyzed, resulting in $\approx$38k SNPs.
The exact amounts can be found in the supplementary materials,
Tables ST3 and ST4.

% \paragraph{Selection of SNPs}
%
SNPs were picked based on ID number, which is linked to their discovery date. To verify that these ID numbers are unrelated to SNP positions, the relative positions of all analyzed SNPs in a protein were determined. This analysis showed no positional bias of the SNPs, as shown in supplementary figure S3.

Per SNP, the \emph{protein} NCBI database was queried for the
protein sequence. For each protein sequence, the protein topology was determined using \verb;PureseqTM;. Using these predicted protein topologies, the SNPs were scored to be located within or outside TMHs.


%%%%%%%%%%%%%%%%%%%%%%%%%%%%%%%%%%%%%%%%%%%%%%%%%%%%%%%%%%%%%%%%%%%%%%%%%%%%%%
\section{Results}
%%%%%%%%%%%%%%%%%%%%%%%%%%%%%%%%%%%%%%%%%%%%%%%%%%%%%%%%%%%%%%%%%%%%%%%%%%%%%%

%%%%%%%%%%%%%%%%%%%%%%%%%%%%%%%%%%%%%%%%%%%%%%%%%%%%%%%%%%%%%%%%%%%%%%%%%%%%%%
\subsection{TMH-derived peptides are predicted to be over-presented in MHC-I}
%%%%%%%%%%%%%%%%%%%%%%%%%%%%%%%%%%%%%%%%%%%%%%%%%%%%%%%%%%%%%%%%%%%%%%%%%%%%%%

Figure \ref{fig:1}A shows the predicted presentation of TMH-derived peptides in MHC-I,
for a human, viral and bacterial proteome.
Per MHC-I allele, it shows the percentage of binders that overlap with a TMH 
with at least one residue.
The horizontal line shows the expected percentage of TMH-derived epitopes 
that would be presented, if TMH-derived epitopes would be presented just as 
likely as epitopes derived from soluble regions,
when assuming equal incidence of soluble and TMH-derived epitope presentation.
For 11 out of 13 MHC-I alleles, TMH-derived epitopes are predicted to be presented more often 
than the null expectation, for a human and bacterial proteome.
For the viral proteome, 12 out of 13 MHC-I alleles present
TMH-derived epitopes more often than expected by chance.
The extent of the over-presentation between the different alleles
is similar for the probed proteomes, 
which strengthens our previous conclusion \citep{bianchi2017} 
that the hydrophobicity of the MHC-binding groove 
is the main factor responsible for the predicted over-presentation 
of TMH-derived peptides.



%%%%%%%%%%%%%%%%%%%%%%%%%%%%%%%%%%%%%%%%%%%%%%%%%%%%%%%%%%%%%%%%%%%%%%%%%%%%%%
\subsection{TMH-derived peptides are predicted to be over-presented in MHC-II}
%%%%%%%%%%%%%%%%%%%%%%%%%%%%%%%%%%%%%%%%%%%%%%%%%%%%%%%%%%%%%%%%%%%%%%%%%%%%%%

We next wondered if the over-representation of TMH-derived peptides would also be present for MHC-II.
Figure \ref{fig:1}A shows the percentages of MHC-II epitopes 
predicted to be overlapping with TMHs for our human, viral and bacterial proteomes.
We found that TMH-derived peptides are over-presented in all
of the 21 MHC-II alleles, 
for a human, bacterial and viral proteome,
except for \verb;HLA-DRB3*0101; in \emph{M. tuberculosis}.
See supplementary Table ST5 
for the exact TMH and epitope counts.

%%%%%%%%%%%%%%%%%%%%%%%%%%%%%%%%%%%%%%%%%%%%%%%%%%%%%%%%%%%%%%%%%%%%%%%%%%%%%%
\subsection{The over-presentation of TMH-derived peptides is caused by the hydrophobicity of the MHC peptide binding groove}
%%%%%%%%%%%%%%%%%%%%%%%%%%%%%%%%%%%%%%%%%%%%%%%%%%%%%%%%%%%%%%%%%%%%%%%%%%%%%%

For MHC-I, we previously showed that the over-presentation of TMH-derived 
peptides is caused by the hydrophobicity of the peptide binding 
grooves \citep{bianchi2017}. 
Figures \ref{fig:1}B and \ref{fig:1}C
show the extent of over-presentation
of TMH-derived epitopes as a function of the hydrophobicity preference score 
for the different human MHC alleles.
An assumed linear correlation explains 88\% of the variability in MHC-I.
For MHC-II, 62\% of the variability is explained by hydrophobicity. 
This indicates that TMH-derived peptides are over-presented, 
because the peptide binding grooves of most MHC-I and -II alleles 
are relatively hydrophobic.

%%%%%%%%%%%%%%%%%%%%%%%%%%%%%%%%%%%%%%%%%%%%%%%%%%%%%%%%%%%%%%%%%%%%%%%%%%%%%%
\subsection{Experimental validation of presentation of TMH-derived peptides}
%%%%%%%%%%%%%%%%%%%%%%%%%%%%%%%%%%%%%%%%%%%%%%%%%%%%%%%%%%%%%%%%%%%%%%%%%%%%%%

The Immune Epitope Database (IEDB) from the National Institutes of Health contains millions of linear epitope sequences obtained
by MHC ligand assays.
For the MHC alleles used in this study, 
we obtained 54,303 and 2,484 linear epitope sequences for the MHC-I
and MHC-II alleles from human origin respectively.
There are relatively few epitopes for MHC-II, 
as MHC-II has many more different alleles than MHC-I,
whereas we selected only the human epitopes found for the 21 MHC-II alleles used in this study.

Figure \ref{fig:2}A and S4 show there are similar levels of
over-presentation of TMH-derived epitopes between (1) the
percentage of TMH-derived epitopes that is reported in the IEDB database
versus (2) the percentage of TMH-derived epitopes that is predicted to be presented
in MHC-I alleles.
For MHC-II alleles, there were too few epitopes per MHC allele
to result in an informative figure. 

In figure \ref{fig:2}B we grouped all the epitopes 
presented by MHC-I and MHC-II alleles 
by the percentage of TMH-derived epitopes,
which are 22\% and 10\%, respectively.

These findings robustly confirm that
epitopes derived from human TMHs are presented in both MHC-I and MHC-II, and support that they are over-presented.
See the supplementary Table ST6 for the exact values.

We also mined the IEDB database for epitopes for
any type of T-cell response from the specified
alleles, from the total reports 36\% and 7\% 
concerned TMH-derived epitopes in MHC class I and II, respectively 
(see Figure S5). 

This data confirms that not only TMH derived epitopes are presented on MHC, 
but this also elicits T-cell mediated immune responses.

%%%%%%%%%%%%%%%%%%%%%%%%%%%%%%%%%%%%%%%%%%%%%%%%%%%%%%%%%%%%%%%%%%%%%%%%%%%%%%%%
\subsection{Human TMHs are evolutionarily conserved}
%%%%%%%%%%%%%%%%%%%%%%%%%%%%%%%%%%%%%%%%%%%%%%%%%%%%%%%%%%%%%%%%%%%%%%%%%%%%%%%%


We addressed the question whether there is an evolutionary advantage in presenting TMHs.
We determined the conservation of TMHs 
by comparing the occurrences of SNPs located in TMHs or soluble protein regions 
for the genes coding for membrane proteins.
We obtained 911 unique gene names associated with the phrase 'membrane protein',
which are genes coding for both membrane-associated proteins (MAPs, which have no TMH) and 
transmembrane proteins (TMPs, which have at least one TMH).
These genes are linked to 4,780 protein isoforms, 
of which 2,553 are predicted to be TMPs and 
2,237 proteins are predicted to be MAPs.
We obtained 37,630 unique variations, 
of which 9,621 are SNPs that resulted in a straightforward amino acids substitution, 
of which 6,062 were located in predicted TMPs.
See supplementary Tables ST3 and ST4 
for the detailed numbers and distributions of SNPs.

Per protein, we calculated two percentages: 
(1) the percentage of a protein sequence length bearing TMHs, 
and (2) the percentage of SNPs located within these predicted TMHs.
Each percentage pair was plotted in figure \ref{fig:3}A.
The proportion of SNPs found in TMHs varied from 
none (i.e., all SNPs were in
soluble regions) to all (i.e., all SNPs were in
TMHs).
To determine if SNPs were randomly distributed over the protein, we performed a linear regression analysis,
and added a 95\% confidence interval on this regression.
This linear fit nearly goes through the origin and has a slope
below the line of equality,
which shows that less SNPs are found in TMHs than expected by chance.

We determined the probability to find the observed amount
of SNPs in TMHs by chance, i.e., when assuming SNPs occur 
just as likely in soluble domains as in TMHs.
We used a binomial Poisson distribution, 
where the number of trails ($n$) equals the number of SNPs, 
which is 21,208. 
The probability of success for the $i$th TMP ($p\_i$), 
is the percentage of residues within a TMH per TMP. 
These percentages are shown as a histogram 
in figure \ref{fig:3}B. 
The expected number of SNPs expected to be found in 
TMHs by chance equals $\sum{p} \approx 4,141$.
As we observed 3,803 SNPs in TMHs, 
we calculated the probability of having that amount or less successes.
We used the type I error cut-off value of $\alpha = 2.5\%$.
The chance to find, within TMHs, this amount or less SNPs 
equals $6.8208 \cdot 10^{-11}$.
We determined the relevance of this finding, by
calculating how much less SNPs are found in TMHs,
when compared to soluble regions, which is the
ratio between the number of SNPs found in TMHs
versus the number of SNPs as expected by chance.
In effect, per 1000 SNPs found in soluble protein domains, 
one finds 918 SNPs in TMHs,
as depicted (as percentages) in figure \ref{fig:3}C. 

We split this analysis for TMPs containing only a single TMH (so-called single-membrane spanners) and TMPs containing multiple TMHs (multi-membrane spanners). 
We hypothesized that single-membrane spanners are less conserved than multi-membrane spanners,
because multi-membrane spanners
might have protein-protein interactions between their TMHs, 
for example to accommodate active sites, and 
thus might have additional structural constraints.
From the split data, we did the same analysis as for the total TMPs.
Figure \ref{fig:4}A 
shows the percentages of TMHs for individual proteins as a function of the
percentage of SNPs located in TMHs.
For both single- and multi-spanners, a linear regression shows that less
SNPs are found in TMHs, than expected by chance.

We also determined the probability to find the 
observed amount of SNPs  by chance in single- and multi-spanners.
For single-spanners, we found 452 SNPs in TMH, where
$\approx462$ were expected by chance. 
The chance to observe this or a lower number by chance is 
$0.319$. As this chance was higher than our $\alpha = 0.025$,
we consider this no significant effect.
For the multi-spanners, we found 3,351 SNPs in TMH, where 
$\approx3,678$ were expected by chance. 
The chance to observe this or a lower number by chance is 
$8.315841 \cdot 10^{-12}$, 
which means this number is significantly less as explained by variation. The TMHs of multi-spanners are thus significantly more conserved than soluble protein regions, whereas this is not the case for single-spanners.

Also, for single- and multi-spanners, we determined the
relevance of this finding by calculating where and how much less SNPs
are found in TMHs when compared to soluble regions, as
depicted in Figure \ref{fig:4}B 
and \ref{fig:4}C.
In effect, per 1,000 SNPs found in soluble protein domains, 
one finds 978 SNPs in TMHs 
of single-spanners
and 911 SNPs in TMHs of multi-spanners.

%%%%%%%%%%%%%%%%%%%%%%%%%%%%%%%%%%%%%%%%%%%%%%%%%%%%%%%%%%%%%%%%%%%%%%%%%%%
\section{Discussion}
%%%%%%%%%%%%%%%%%%%%%%%%%%%%%%%%%%%%%%%%%%%%%%%%%%%%%%%%%%%%%%%%%%%%%%%%%%%

% \paragraph{General}

Epitope prediction is important to understand the immune system function
and for the design of vaccines.
In this study, we provide evidence that epitopes
derived from TMHs are a major but overlooked source of MHC epitopes. 
Our bioinformatics predictions indicate that the TMH-derived epitope repertoire is larger than expected by chance for both MHC-I and MHC-II, regardless of the organism. Moreover, reanalysis of MHC-ligands from the IEDB database confirmed the presentation of TMH-derived epitopes. Therefore, it seems likely that TMH-derived epitopes would also result in enhanced T cell responses, although the conservation of TMHs might promote the deletion of T cells responsive to TMH-derived epitopes by central tolerance mechanisms. Finally, our SNP analysis shows that TMHs are evolutionary more conserved 
than solvent-exposed protein regions.

\subsection{Mechanism of MHC presentation of TMH-derived epitopes}
% \paragraph{TMH presentation}

Although our data show that 
TMH-derived epitopes are presented in all classical MHC-I and MHC-II alleles, 
the molecular mechanisms of how integral membrane proteins are processed 
for MHC presentation are largely unknown \citep{bianchi2017}. 
Most prominently, the fundamental principles of 
how TMHs are extracted from their hydrophobic lipid environments 
into the aqueous vacuolar lumen, 
leading to subsequent proteolytic processing are unresolved. 

A first possibility is that the extraction of TMPs from the membrane 
is mediated by the ER-associated degradation (ERAD) machinery. 
For MHC class I (MHC-I) antigen presentation of soluble proteins, 
the loading of the epitope primarily occurs at the endoplasmatic reticulum (ER). 
The chaperones tapasin (TAPBP), ERp57 (PDIA3), 
and calreticulin (CALR) \citep{rock2016present} first assemble 
and stabilize the heavy and light chains of MHC-I. 
Later, this complex binds to the transporter 
associated with antigen processing (TAP) 
leading to the formation of the so-called peptide-loading complex (PLC). 
The PLC drives import of peptides into the ER 
and mediates their subsequent loading into the peptide-binding groove of MHC-I \citep{blees2017structure}. 
Membrane proteins first will have to be extracted from the membrane 
before they become amenable to this MHC-I loading by the PLC. 
In the ER, this process can be orchestrated by the ERAD machinery, 
consisting of several chaperones that recognize TMPs, 
ubiquitinate them, and extract them from the ER membrane 
into the cytosol (retrotranslocation) for proteasomal degradation \citep{preston2017evolving,meusser2005erad}. 
Similar to the peptides generated from soluble proteins, 
the TMP-derived peptides might then be re-imported by TAP 
into the ER for MHC-I loading. 
This ERAD-driven antigen retrotranslocation might be facilitated by lipid bodies (LBs) \citep{bougneres2009role}, 
since LBs can serve as cytosolic sites for ubiquitination of ER-derived cargo \citep{fujimoto2006proteasomal}. 

A second possibility is that TMPs are proteolytically processed 
by intramembrane proteases that cleave TMHs while they are still membrane embedded. Supporting this hypothesis is the well-established notion that peptides generated by signal peptide peptidases (SPPs), an important class of intramembrane proteases that cleave TMH-like signal sequences, 
are presented on a specialized class of MHC-I called HLA-E \citep{oliveira2015alternative}. 
The loading of peptides generated by SPP onto MHC-I does not depend on the proteosome and TAP, 
possibly because the peptides are directly released into the lumen of the ER \citep{oliveira2015alternative}. 
However, this mechanism cannot explain how most membrane proteins can be processed for antigen presentation, because SPPs only cleave TMH-like signal sequences at their C-termini, and N-terminal domains will hence not be removed. 
Nevertheless, the presentation of peptides with a high hydrophobicity index 
was shown to be independent of TAP as well \citep{lautscham2001processing}, 
suggesting that the TMH peptides might perhaps be released directly in the ER lumen by other intramembrane proteases. 

A third possibility is that peptide processing and MHC-loading occur in multivesicular bodies (MVBs) \citep{oliveira2015alternative}. 
TMPs can be routed from the plasma membrane and other organelles by vesicular trafficking to endosomes. Eventually, these TMPs can be sorted by the endosomal sorting complexes required for transport (ESCRT) pathway into luminal invaginations that pinch off from the limiting membrane and form intraluminal vesicles. This thus results in MVBs where the membrane proteins destined for degradation are located in intraluminal vesicles. Upon the fusion of MVBs with lysosomes, 
the entire intraluminal vesicles including the TMPs are degraded \citep{gruenberg2020life}. 
Via this mechanism, TMPs might well be processed for antigen presentation, 
particularly since the loading of MHC-II molecules is well understood 
to occur in MVBs \citep{kleijmeer2001reorganization,peters1991segregation,zwart2005spatial}. 
However, such processing of membrane proteins in MVBs for antigen presentation poses a problem, because complexes of HLA-DR with its antigen-loading chaperon HLA-DM were only observed on intraluminal vesicles, 
but not on the limiting membranes of MVBs \citep{zwart2005spatial}, 
indicating that epitope loading of MHC-II also occurs at intraluminal vesicles. This observation hence raises the question how the intraluminal vesicles carrying the TMPs destined for antigen presentation can be selectively degraded, while the intraluminal vesicles carrying the MHC-II remain intact. A second problem is that phagosomes carrying internalized microbes lack intraluminal vesicles, 
and it is hence unclear how TMPs from these microbes 
would be routed to MVBs for MHC-II loading \citep{zwart2005spatial}.

Alternatively to the enzymatic degradation of lipids in MVBs by lipases \citep{sander2016lipase,gilleron2016lysosomal}, 
they might be oxidatively degraded by reactions with radical oxygen species produced by the NADPH oxidase NOX2 \citep{dingjan2016lipid}. 
This oxidation can result in a destabilization and disruption of membranes \citep{dingjan2016lipid} 
and might thereby lead to the extraction of TMPs. 
Due to the hydrophobic nature of TMHs, 
however, the extracted proteins will likely aggregate 
and it is unclear how these aggregates would be processed further for MHC loading. 


%%%%%%%%%%%%%%%%%%%%%%%%%%%%%%%%%%%%%%%%%%%%%%%%%%%%%%%%%%%%%%%%%%%%%%%%%%%
\subsection{Evolutionary conservation of TMHs}
%%%%%%%%%%%%%%%%%%%%%%%%%%%%%%%%%%%%%%%%%%%%%%%%%%%%%%%%%%%%%%%%%%%%%%%%%%%

% \paragraph{Selection undetectable in whole proteome}
 
In general, one might expect that evolutionary selection shapes an immune system 
where surveillance is directed towards protein regions 
essential for the survival, proliferation and/or virulence or pathogenic microbes, 
as these will be most conserved.
In SARS-CoV-2, for example, there is preliminary evidence that the strongest
selection pressure is directed upon residues that change its 
virulence \citep{velazquez2020positive}.
These regions, however, may only account for a small part of a pathogen's proteome.
Additionally, the structure and function of these essential regions might differ widely between different pathogenic proteins.
Because of this scarcity and variance in targets, 
one can imagine that it will be mostly unfeasible 
to provide innate immune responses against such rare essential protein regions, 
as suggested in a study on influenza \citep{han2019individual},
where it was found that the selection pressure
exerted by the immune system was either weak or absent.
 
% \paragraph{Selection may be detectable in TMHs}

Evolutionary selection of pathogens by a host's immune system,
however, is more likely to occur for protein patterns that are general,
over patterns that are rare.
While essential catalytic sites in a pathogenic proteome
might be relatively rare, TMHs are common and thus might be a more feasible 
target for evolution to respond to.
Indeed, we have found the signature of evolution when both factors,
that is, TMHs and catalytic sites are likely to co-occur,
which is in TMPs that span the membrane at least twice.
In contrast to single-spanners, where we found no significant evolutionary conservation, 
the TMHs of multi-spanners are more evolutionary conserved than soluble protein regions. 
Likely, the TMHs in many multi-spanners need to interact which each other 
for correct protein structure and function 
and they might hence be more structurally constrained 
compared to the TMHs of single-spanners.
Thus, we speculate that the human immune system is more attentive 
towards TMHs in multi-spanners, as these are evolutionarily more conserved. 

There have been more efforts to assess the conservation of TMHs,
using different methodologies.
One such example is a study by Stevens and Arkin \citep{stevens2001substitution}, 
in which aligned protein sequence data was used.
Also this study found that TMHs are evolutionarily more conserved,
as the mean amino acid substitution rate in TMHs is about ten
percent lower,
which is a similar value as we found.
Another example is a study by Oberai, et al. \citep{oberai2009structural} that estimated the conservation
scores for TMHs and soluble regions based on 
alignments of evolutionary related proteins,
and also found that TMHs are more conserved, 
with a conservation score that was 17\% higher in 
TMHs.
Note that the last study also found that mutations in human TMHs are likelier to cause
a disease, in line with our conclusion that TMHs are more conserved.

% \paragraph{Conclusion}

Together, from this study, two important conclusions can be drawn. 
First, the MHC over-presentation of TMHs is likely a general feature 
and predicted to occur for most alleles of both MHC-I and -II 
and for humans as well as bacterial and viral pathogens. 
Second, TMHs are genuinely more evolutionary conserved than soluble protein motifs, 
at least in the human proteome. 

%
% Frontiers text, just keep it in for now to be sure.
%
%\section{Article types}
%
% For requirements for a specific article type please refer to the Article Types on any Frontiers journal page. Please also refer to  \href{http://home.frontiersin.org/about/author-guidelines#Sections}{Author Guidelines} for further information on how to organize your manuscript in the required sections or their equivalents for your field
%
% For Original Research articles, please note that the Material and Methods section can be placed in any of the following ways: before Results, before Discussion or after Discussion.
%
%\section{Manuscript Formatting}
%
%\subsection{Heading Levels}
%
%There are 5 heading levels
%
%\subsection{Level 2}
%\subsubsection{Level 3}
%\paragraph{Level 4}
%\subparagraph{Level 5}
%
%\subsection{Equations}
%Equations should be inserted in editable format from the equation editor.
%
%\begin{equation}
%\sum x+ y =Z\label{eq:01}
%\end{equation}
%
%\subsection{Figures}
%Frontiers requires figures to be submitted individually, in the same order as they are referred to in the manuscript. Figures will then be automatically embedded at the bottom of the submitted manuscript. Kindly ensure that each table and figure is mentioned in the text and in numerical order. Figures must be of sufficient resolution for publication \href{http://home.frontiersin.org/about/author-guidelines#ResolutionRequirements}{see here for examples and minimum requirements}. Figures which are not according to the guidelines will cause substantial delay during the production process. Please see \href{http://home.frontiersin.org/about/author-guidelines#GeneralStyleGuidelinesforFigures}{here} for full figure guidelines. Cite figures with subfigures as figure \ref{fig:2}B.
%
%
%\subsubsection{Permission to Reuse and Copyright}
%Figures, tables, and images will be published under a Creative Commons CC-BY licence and permission must be obtained for use of copyrighted material from other sources (including re-published/adapted/modified/partial figures and images from the internet). It is the responsibility of the authors to acquire the licenses, to follow any citation instructions requested by third-party rights holders, and cover any supplementary charges.
%%Figures, tables, and images will be published under a Creative Commons CC-BY licence and permission must be obtained for use of copyrighted material from other sources (including re-published/adapted/modified/partial figures and images from the internet). It is the responsibility of the authors to acquire the licenses, to follow any citation instructions requested by third-party rights holders, and cover any supplementary charges.
%
%\subsection{Tables}
%Tables should be inserted at the end of the manuscript. Please build your table directly in LaTeX.Tables provided as jpeg/tiff files will not be accepted. Please note that very large tables (covering several pages) cannot be included in the final PDF for reasons of space. These tables will be published as \href{http://home.frontiersin.org/about/author-guidelines#SupplementaryMaterial}{Supplementary Material} on the online article page at the time of acceptance. The author will be notified during the typesetting of the final article if this is the case. 
%
%\section{Nomenclature}
%
%\subsection{Resource Identification Initiative}
%To take part in the Resource Identification Initiative, please use the corresponding catalog number and RRID in your current manuscript. For more information about the project and for steps on how to search for an RRID, please click \href{http://www.frontiersin.org/files/pdf/letter_to_author.pdf}{here}.
%
%\subsection{Life Science Identifiers}
%Life Science Identifiers (LSIDs) for ZOOBANK registered names or nomenclatural acts should be listed in the manuscript before the keywords. For more information on LSIDs please see \href{http://www.frontiersin.org/about/AuthorGuidelines#InclusionofZoologicalNomenclature}{Inclusion of Zoological Nomenclature} section of the guidelines.
%
%
%\section{Additional Requirements}
%
%For additional requirements for specific article types and further information please refer to \href{http://www.frontiersin.org/about/AuthorGuidelines#AdditionalRequirements}{Author Guidelines}.
%

\section*{Conflict of Interest Statement}

%All financial, commercial or other relationships that might be perceived by the academic community as representing a potential conflict of interest must be disclosed. If no such relationship exists, authors will be asked to confirm the following statement: 

The authors declare that the research was conducted in the absence of any commercial or financial relationships that could be construed as a potential conflict of interest.

\section*{Author Contributions}

% The Author Contributions section is mandatory for all articles, including articles by sole authors. If an appropriate statement is not provided on submission, a standard one will be inserted during the production process. The Author Contributions statement must describe the contributions of individual authors referred to by their initials and, in doing so, all authors agree to be accountable for the content of the work. Please see  \href{http://home.frontiersin.org/about/author-guidelines#AuthorandContributors}{here} for full authorship criteria.

RJCB and FB conceived the idea for this research. 
MVB helped with the proteome analysis of \emph{M. tuberculosis}.
RJCB wrote the code.
RJCB, MB, GvdB and FB wrote the article.

\section*{Funding}

% Details of all funding sources should be provided, including grant numbers if applicable. Please ensure to add all necessary funding information, as after publication this is no longer possible.

FB is funded by a Veni grant from the Netherlands Organization for Scientific
Research (016.Veni.192.026) and an Off-Road Grant from the Dutch Medical Science Foundation (ZonMW 04510011910005).
GvdB is funded by a Young Investigator Grant from 
the Human Frontier Science Program (HFSP; RGY0080/2018), 
and a Vidi grant from 
the Netherlands Organization for Scientific Research (NWO-ALW VIDI 864.14.001). 
GvdB has received funding from the European Research Council (ERC) 
under the European Union’s Horizon 2020 research and 
innovation programme (grant agreement No. 862137. 

\section*{Acknowledgments}

% This is a short text to acknowledge the contributions of specific colleagues, institutions, or agencies that aided the efforts of the authors.

We thank the Center for Information Technology of the University 
of Groningen for its support and for providing access to the Peregrine 
high performance computing cluster. 

\section*{Supplemental Data}
 \href{http://home.frontiersin.org/about/author-guidelines#SupplementaryMaterial}{Supplementary Material} should be uploaded separately on submission, if there are Supplementary Figures, please include the caption in the same file as the figure. LaTeX Supplementary Material templates can be found in the Frontiers LaTeX folder.

\section*{Data Availability Statement}
The data presented in the study are deposited in the Zenodo repository, accession number 10.5281/zenodo.5809139.
Additionally, all code, intermediate and final results are archived at 
\url{https://github.com/richelbilderbeek/bbbq_article}.

\bibliographystyle{frontiersinSCNS_ENG_HUMS} % for Science, Engineering and Humanities and Social Sciences articles, for Humanities and Social Sciences articles please include page numbers in the in-text citations
%\bibliographystyle{frontiersinHLTH&FPHY} % for Health, Physics and Mathematics articles
\bibliography{bbbq_article}

%%% Make sure to upload the bib file along with the tex file and PDF
%%% Please see the test.bib file for some examples of references

\section*{Figure captions}

%%% Please be aware that for original research articles we only permit a combined number of 15 figures and tables, one figure with multiple subfigures will count as only one figure.
%%% Use this if adding the figures directly in the mansucript, if so, please remember to also upload the files when submitting your article
%%% There is no need for adding the file termination, as long as you indicate where the file is saved. In the examples below the files (logo1.eps and logos.eps) are in the Frontiers LaTeX folder
%%% If using *.tif files convert them to .jpg or .png
%%%  NB logo1.eps is required in the path in order to correctly compile front page header %%%

%
% Figure 1
%
\begin{figure}[h!]
\begin{center}
\includegraphics[width=0.8\textwidth]{bbbq_article_figures_Page_1}% This is a *.eps file
\end{center}
\caption{
    \textbf{Over-presentation of TMH-derived epitopes on most MHC-I and -II alleles}
    \textbf{(A)} 
    The percentage of epitopes for MHC-I and -II alleles that are predicted to 
    overlap with TMHs for the proteomes of SARS-CoV-2 (top row), human (middle 
    row) and \emph{M. tuberculosis} (MtB; bottom row).
    The pair of horizontal red lines in each plot indicate the lower and upper bound 
    of the 99\% confidence interval.
    See supplementary Tables ST5 and ST7
    for the exact TMH and  epitope counts.
    \textbf{(B-C)}
    Correlation between the percentages of predicted TMH-derived epitopes
    and the hydrophobicity score of all predicted epitopes for 
    human MHC-I \textbf{(B)} and MHC-II alleles \textbf{(C)}.
    Diagonal red line: linear regression analysis. 
    Labels are shorthand for the HLA alleles,
    see the supplementary Table ST8 for the names.}
  \label{fig:1}
\end{figure}
% 1A: \label{fig:bbbq_1_smart_results}
% 1B: \label{fig:hydrophobicity_1}
% 1C: \label{fig:hydrophobicity_2}

%
% Figure 2
%
\begin{figure}[h!]
\begin{center}
\includegraphics[width=\textwidth]{bbbq_article_figures_Page_2}% This is a *.eps file
\end{center}
\caption{
    \textbf{
      Analysis of epitope database shows that TMH derived epitopes are over presented.
    }
    The percentage of epitopes for MHC-I and -II alleles that overlap with TMHs
    that are presented. The pair of horizontal red lines in each plot indicate the
    lower and upper bound of the 99\% confidence interval. Note that only one line
    is visible as this interval is relatively narrow. 
    Alleles are listed in Table ST8).
    \textbf{(A)} 
    Observed and predicted percentage of TMH-derived epitopes for MHC-I alleles. 
    \textbf{(B)} 
    MHC ligands from IEDB corresponding to TMH-derived epitopes. 
    The numbers above the bars denotes the number of TMH derived epitopes obtained.
}
\label{fig:2}
\end{figure}
% Figure 2a: \label{fig:2a}
% Figure 2b: \label{fig:2b}



%
% Figure 3
%
\begin{figure}[h!]
\begin{center}
\includegraphics[width=\textwidth]{bbbq_article_figures_Page_3}% This is a *.eps file
\end{center}
\caption{
    \textbf{Evolutionary conservation of human TMHs.}
    \textbf{(A)} 
    Percentage of SNPs found in TMHs.
    Each point shows for one protein the predicted percentage of
    amino acids that are part of a TMH ($x$-axis) and the observed occurrence of SNPs being located
    within a TMH ($y$-axis).
    The dashed diagonal line shows the line of equality (i.e.,
    equal conservation of TMHs and soluble protein regions). 
    The diagonal red line indicates a linear fit, 
    the gray area its 95\% confidence interval.
    \textbf{(B)}
    Distribution of the percentages of TMH in the TMPs used in this study.
    \textbf{(C)}
    The number of SNPs in TMHs as expected by chance (left bar) 
    and found in the dbSNP database (right bar).
    Percentages show the relative conservation
    of SNPs in TMHs found relative to stochastic chance.
}
\label{fig:3}
\end{figure}
% Figure 3A: \label{fig:f_snps_found_and_expected}
% Figure 3B: \label{fig:f_tmh_ncbi}
% Figure 3C: \label{fig:conservation}


%
% Figure 4
%
\begin{figure}[h!]
\begin{center}
\includegraphics[width=\textwidth]{bbbq_article_figures_Page_4}% This is a *.eps file
\end{center}
\caption{\textbf{Membrane proteins with multiple TMHs are evolutionary more conserved than proteins with only a single TMH.}
      \textbf{(A)} 
      Percentage of SNPs found in TMPs predicted to have only a single
      (left) or multiple (right) TMHs.
      Each point shows for one protein the predicted percentage of amino acids that are part of a TMH ($x$-axis) and the observed occurrence of SNPs being located within a TMH ($y$-axis). The dashed diagonal lines show the line of equality (i.e., equal conservation of TMHs and soluble protein regions). 
      The diagonal red lines indicate a linear fit, the gray areas their 95\% confidence intervals.
      \textbf{(B)} 
      The number of SNPs in TMHs as expected by chance 
      and observed in the dbSNP database, 
      for TMPs with one TMH (single-spanners) and multiple TMHs (multi-spanners).
      Percentages show the relative conservation
      of SNPs in TMHs found relative to the stochastic chances.
      \textbf{(C)} 
      Distribution of the proportion of amino acids residing
      in the plasma membrane. 
}
\label{fig:4}
\end{figure}
% Figure 4A: \label{fig:f_snps_found_and_expected_per_spanner}
% Figure 4B: \label{fig:conservation_per_spanner}
% Figure 4C: \label{fig:f_tmh_ncbi_per_spanner}

%%% If you are submitting a figure with subfigures please combine these into one image file with part labels integrated.
%%% If you don't add the figures in the LaTeX files, please upload them when submitting the article.
%%% Frontiers will add the figures at the end of the provisional pdf automatically
%%% The use of LaTeX coding to draw Diagrams/Figures/Structures should be avoided. They should be external callouts including graphics.

\end{document}
