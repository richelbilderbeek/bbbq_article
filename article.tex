\documentclass{article}

% Bibliography
\usepackage{natbib}
\bibpunct{(}{)}{;}{a}{}{;}

% Use 'It was found that A is B (Name 1234)' style
\setcitestyle{authoryear,open={},close={}}

% Affiliations
\usepackage{authblk}
\title{
  Transmembrane helices are also 
  an overlooked source of major histocompatibility complex class II epitopes
  and evolutionary more conserved than expected by chance
}
\author[1]{Rich\`el J.C. Bilderbeek}
\author[1]{?Rampal S. Etienne}
\author[2]{Frans Bianchi}
\affil[1]{Groningen Institute for Evolutionary Life Sciences, University of Groningen, Groningen, The Netherlands}
\affil[2]{Frans' Institute, University of Groningen, Groningen, The Netherlands}

% Use double spacing
\usepackage{setspace}
\doublespacing

\usepackage{listings}
\usepackage{hyperref}
\usepackage{todonotes}
\usepackage{verbatim}
\usepackage{pgf}
\usepackage{bm}
\usepackage{multirow}
\usepackage{amsfonts}
\usepackage{array}
\usepackage{array}
\usepackage{booktabs}
\newcolumntype{C}[1]{>{\centering\arraybackslash}p{#1}}
\newcolumntype{L}[1]{>{\raggedright\arraybackslash}p{#1}}
\usepackage{longtable}

\usepackage{tkz-graph}
\usetikzlibrary{arrows,automata}
\usetikzlibrary{calc}
\usetikzlibrary{arrows.meta}

% sidewaysfigure
\usepackage{rotating}

% Style of listings
% From http://r.789695.n4.nabble.com/How-to-nicely-display-R-code-with-the-LaTeX-package-listings-tp4648110.html
\usepackage{fancyvrb} 
\definecolor{codegreen}{rgb}{0,0.6,0}
\definecolor{codegray}{rgb}{0.5,0.5,0.5}
\definecolor{codepurple}{rgb}{0.58,0,0.82}
\definecolor{backcolor}{rgb}{0.95,0.95,0.92}
\lstdefinestyle{mystyle}{
  language=R,% set programming language
  basicstyle=\ttfamily\small,% basic font style
  commentstyle=\color{gray},% comment style
  % numbers=left,% display line numbers on the left side
  numberstyle=\scriptsize,% use small line numbers
  numbersep=10pt,% space between line numbers and code
  tabsize=2,% sizes of tabs
  showstringspaces=false,% do not replace spaces in strings by a certain character
  captionpos=b,% positioning of the caption below
  breaklines=true,% automatic line breaking
  escapeinside={(*}{*)},% escaping to LaTeX
  fancyvrb=true,% verbatim code is typset by listings
  extendedchars=false,% prohibit extended chars (chars of codes 128--255)
  alsoletter={.<-},% becomes a letter
  alsoother={$},% becomes other
  otherkeywords={!=, ~, $, \&, \%/\%, \%*\%, \%\%, <-, <<-, /},% other keywords
  deletekeywords={c}% remove keywords 
}
\lstset{style=mystyle}

% Adds numbered lines
\usepackage{lineno}
\linenumbers

% Rename 'Abstract' to 'Summary 
\usepackage[english]{babel}
\addto{\captionsenglish}{\renewcommand{\abstractname}{Summary}}

%comments
\newcommand{\frans}[1]{\textcolor{blue}{\textbf{[FB: #1]}}}
\newcommand{\richel}[1]{\textcolor{orange}{\textbf{[RB: #1]}}}
\newcommand{\rampal}[1]{\textcolor{green}{\textbf{[RSE: #1]}}}


\begin{document}

\maketitle

\begin{abstract}

Transmembrane helices (TMHs) are an overlooked source of 
major histocompatibility complex (MHC) class I epitopes.
It is unknown why this is the case and 
whether MHC-II also is an overlooked source of epitopes.
This study investigates if there is an evolutionary conservation 
of a pathogen's TMHs and if this is matches 
with MHC-II's over/equally/under-presenting TMHs.
We find that ...
\end{abstract}

{\bf Keywords:} antigen presentation, membrane proteins, bioinformatics, 
adaptive immunity, transmembrane domain, epitopes, T lymphocyte, MHC-2

%%%%%%%%%%%%%%%%%%%%%%%%%%%%%%%%%%%%%%%%%%%%%%%%%%%%%%%%%%%%%%%%%%%%%%%%%%%%%%%%
\section{Introduction}
%%%%%%%%%%%%%%%%%%%%%%%%%%%%%%%%%%%%%%%%%%%%%%%%%%%%%%%%%%%%%%%%%%%%%%%%%%%%%%%%

For MHC-1, we know that epitopes derived from transmembrane helices (TMHs) are 
over-presented by all human leukocyte antigen (HLA)-A and most HLA-B super 
types [\cite{bianchi2017transmembrane}].

A hypothesis offered by that same paper is that the presentation of 
especially TMHs may have an evolutionary advantage for the (human) host.
The argumentation given is that the presentation of parts of TMHs
is likelier to detect pathogenic proteins, as TMHs are already conserved
due to the functional requirement of being able to span a membrane.
Additionally, there are fewer escape mutations possible for the pathogen,
as relatively less of these will result in a functional TMH.

However, that argumentation can also go the other way: as there are less
polypeptide fragments from (the less amount of) possible TMHs, 
it is likelier that host and pathogen
share the same protein sequence, 
resulting the pathogen avoiding detection.

Within the immune response, MHC-I is unique in [doing something].
The other most important participant in the immune response in MHC-II,
which is more involved in detecting bacterial pathogens, 
e.g. tuberculosis, [and others].

From the earlier research done on MHC-I, it naturally follows to check
if MHC-II are likelier to present TMHs than expected by chance.

Extending the previous study on MHC-I, we also measure the evolutionary
conservation of TMH epitopes in Mycoplasma pathogens.

%%%%%%%%%%%%%%%%%%%%%%%%%%%%%%%%%%%%%%%%%%%%%%%%%%%%%%%%%%%%%%%%%%%%%%%%%%%%%%%%
\section{Methods}
%%%%%%%%%%%%%%%%%%%%%%%%%%%%%%%%%%%%%%%%%%%%%%%%%%%%%%%%%%%%%%%%%%%%%%%%%%%%%%%%

Transmembrane helices and strong MHC-II-binding peptides
were predicted for a tuberculosis reference proteome 
from \url{https://www.ebi.ac.uk/reference_proteomes}
using the \verb;epitopeome; 
\richel{
  I'd enjoy a better name
}
R package [\cite{epitopeome}].
We picked the three MHC-II alleles that are most abundant 
in the current human population, 
which are DRB4*0101, HLA-DPA10103-DPB10402 
\richel{
  We decided for DPB1*0402, but there are multiple alleles matching.
  I just picked the first one of the alleles matching, which are:
  HLA-DPA10x-DPB10402, where x equals 103-110, 201-204, 301-303 or 401
}
and DQA1*0501/DQB1*0301 \richel{reference here}.
The 5\% peptides with the lowest IC50 values were defined as binders.
We then simply counted the number of amino acids that were present inside the
cell, within the membrane or outside of the cell, as well as if it was part 
of a strong MHC-II binding site, as shown in figue \ref{tab:results}.


The \verb;epitopeome; R package \cite{epitopeome} binds \richel{pun intended} 
together the \verb;tmhmm; [\cite{tmhmm}] and \verb;netmhc2pan; [\cite{netmhc2pan}] 
R packages. \verb;tmhmm; provides an R interface to
TMHMM [\cite{krogh2001predicting}, \cite{sonnhammer1998hidden}], a tool
to predict where membrane proteins' amino acids are located within the
membrance.   
\verb;netmhc2pan; provides an R interface to
NetMHC2pan [\cite{jensen2018improved}], a tool
to predict MHC-II binding to proteins.

The evolutionary conservation of TMHs was measured from a
DNA alignment of multiple transmembrane proteins in multitiple
species of the mycoplasma bacterial family
\richel{how obtained exactly? how alignment done?}.

Using the \verb;tmhprot; R package [\cite{tmhprot}], the DNA alignment is split
into two alignments, one for the TMH parts, another for the non-TMH
parts. Each alignment was tested by the \verb;mcbette; 
R package [\cite{mcbette}] to select the Bayesian inference model with
the highest evidence (a.k.a. the marginal likelihood) using the nested
sampling approach as described in \cite{maturana2018model},
using the popular Bayesian phylogenetic tool 
BEAST2 [\cite{bouckaert2014beast}] in the back-end.
\verb;mcbette; used a set of 40 candidate inference models, 
consisting of all combinations of 
4 site models (JC, HKY, TN, GTR), 
2 clock models (strict, RLN) and 
5 tree priors (Yule, BD, CBS, CCP, CEP).

For each alignment, the inference model with the highest evidence
is used in a Bayesian inference. From each Bayesian posterior,
the parameter estimates regarding mutation rates (including clock rate)
were obtained and compared to determine which realized mutation rate is lower.
The Markov chain Monte Carlo was set up in such a way that the effective sample
size for the likelihood of the inference model is above the recommended value
of 200 [\cite{bouckaert2014beast}].

%%%%%%%%%%%%%%%%%%%%%%%%%%%%%%%%%%%%%%%%%%%%%%%%%%%%%%%%%%%%%%%%%%%%%%%%%%%%%%%%
\section{Results}
%%%%%%%%%%%%%%%%%%%%%%%%%%%%%%%%%%%%%%%%%%%%%%%%%%%%%%%%%%%%%%%%%%%%%%%%%%%%%%%%

Table \ref{tab:results} shows the location and binding strength for the
tuberculosis proteome.

\input{table_imoimo.latex}
% has label tab:results
\richel{
  I'd enjoy 
  (1) a row with 'expected by chance', 
  (2) using percentages instead,
  (3) merging inside and outside
}

The inference model with the highest evidence in the
TMH-only alignment was [yet unknown] and [also yet unknown]
for the non-TMH alignment. Individual model weights are shown
in tables \ref{tab:evidences_tmh} 
and \ref{tab:evidences_non_tmh}.

The Bayesian inference resulted in [a distribution of mutation rates],
as shown in [absent figure].
The ESSes of the Bayesian parameter estimates was above 200, exact values
are shown in tables \ref{tab:esses_tmh} and \ref{tab:esses_non_tmh}.

%%%%%%%%%%%%%%%%%%%%%%%%%%%%%%%%%%%%%%%%%%%%%%%%%%%%%%%%%%%%%%%%%%%%%%%%%%%%%%%%
\section{Conclusion}
%%%%%%%%%%%%%%%%%%%%%%%%%%%%%%%%%%%%%%%%%%%%%%%%%%%%%%%%%%%%%%%%%%%%%%%%%%%%%%%%

We conclude that MHC-II binds to TMH peptides with a higher/lower/equal
probability than expected by chance. 

We conclude that the evolutionary conservation if the TMH parts of membrane
proteins is higher/less/equal compare to its non-TMH counterparts.

%%%%%%%%%%%%%%%%%%%%%%%%%%%%%%%%%%%%%%%%%%%%%%%%%%%%%%%%%%%%%%%%%%%%%%%%%%%%%%%%
\section{Discussion}
%%%%%%%%%%%%%%%%%%%%%%%%%%%%%%%%%%%%%%%%%%%%%%%%%%%%%%%%%%%%%%%%%%%%%%%%%%%%%%%%

We compared the mutation rates between the TMH and non-TMH part of
multiple mycoplasma species. Where we expect no variation 
in mutation rate for every TMH amino acid,
\richel{
  we can test this, but unsure if that would make sense
}
we know that non-TMH part will have regions of different evolutionary
conservation: functional domains, especially in protein-protein
interactions will be strongly conserved, due to an even more constrained
set of peptides that enable a certain function.


%%%%%%%%%%%%%%%%%%%%%%%%%%%%%%%%%%%%%%%%%%%%%%%%%%%%%%%%%%%%%%%%%%%%%%%%%%%%%%%%
\section{Acknowledgements}
%%%%%%%%%%%%%%%%%%%%%%%%%%%%%%%%%%%%%%%%%%%%%%%%%%%%%%%%%%%%%%%%%%%%%%%%%%%%%%%%

Thanks to Geert van den Bogaart for his help \richel{reword}.
Thanks to Maxim for his help \richel{reword}.
We thank the Netherlands 
Organization for Scientific Research (NWO) for financial support 
through a VICI grant awarded to RSE.

%%%%%%%%%%%%%%%%%%%%%%%%%%%%%%%%%%%%%%%%%%%%%%%%%%%%%%%%%%%%%%%%%%%%%%%%%%%%%%%%
\section{Data Accessibility}
%%%%%%%%%%%%%%%%%%%%%%%%%%%%%%%%%%%%%%%%%%%%%%%%%%%%%%%%%%%%%%%%%%%%%%%%%%%%%%%%

All code is archived at \url{http://github.com/richelbilderbeek/someplace},
with DOI \url{https://doi.org/12.3456/zenodo.1234567}.

%%%%%%%%%%%%%%%%%%%%%%%%%%%%%%%%%%%%%%%%%%%%%%%%%%%%%%%%%%%%%%%%%%%%%%%%%%%%%%%%
\section{Authors' contributions}
%%%%%%%%%%%%%%%%%%%%%%%%%%%%%%%%%%%%%%%%%%%%%%%%%%%%%%%%%%%%%%%%%%%%%%%%%%%%%%%%

RJCB and FB conceived the idea for this research. 
RJCB wrote the code.
FB supplied the DNA alignment.
RJCB, FB and RSE contributed to writing the article.

%%%%%%%%%%%%%%%%%%%%%%%%%%%%%%%%%%%%%%%%%%%%%%%%%%%%%%%%%%%%%%%%%%%%%%%%%%%%%%%%
% Bibliography
%%%%%%%%%%%%%%%%%%%%%%%%%%%%%%%%%%%%%%%%%%%%%%%%%%%%%%%%%%%%%%%%%%%%%%%%%%%%%%%%
% MEE style
\bibliographystyle{mee}
\bibliography{article}
%%%%%%%%%%%%%%%%%%%%%%%%%%%%%%%%%%%%%%%%%%%%%%%%%%%%%%%%%%%%%%%%%%%%%%%%%%%%%%%%

\appendix

%%%%%%%%%%%%%%%%%%%%%%%%%%%%%%%%%%%%%%%%%%%%%%%%%%%%%%%%%%%%%%%%%%%%%%%%%%%%%%%%
\section{Model selection}
%%%%%%%%%%%%%%%%%%%%%%%%%%%%%%%%%%%%%%%%%%%%%%%%%%%%%%%%%%%%%%%%%%%%%%%%%%%%%%%%

\begin{table}[ht]
\centering
\begin{tabular}{rlllrr}
  \hline
 & Site model & Clock model & Tree prior & log(evidence) & Weight \\ 
  \hline
1 & JC & Strict & Yule & -8201.109 & 0.002 \\ 
  2 & JC & Strict & BD & -8203.346 & 0.000 \\ 
  3 & JC & Strict & CCP & -8195.927 & 0.278 \\ 
  4 & JC & Strict & CEP & -8194.976 & 0.720 \\ 
   \hline
\end{tabular}
\caption{
  Evidences for the TMH-only alignment
  \richel{These are just example values}
} 
\label{tab:evidences_tmh}
\end{table}

\begin{table}[ht]
\centering
\begin{tabular}{rlllrr}
  \hline
 & Site model & Clock model & Tree prior & log(evidence) & Weight \\ 
  \hline
1 & JC & Strict & Yule & -8201.109 & 0.002 \\ 
  2 & JC & Strict & BD & -8203.346 & 0.000 \\ 
  3 & JC & Strict & CCP & -8195.927 & 0.278 \\ 
  4 & JC & Strict & CEP & -8194.976 & 0.720 \\ 
   \hline
\end{tabular}
\caption{
  Evidences for the non-TMH alignment
  \richel{These are just example values}
} 
\label{tab:evidences_non_tmh}
\end{table}

%%%%%%%%%%%%%%%%%%%%%%%%%%%%%%%%%%%%%%%%%%%%%%%%%%%%%%%%%%%%%%%%%%%%%%%%%%%%%%%%
\section{Inference strength}
%%%%%%%%%%%%%%%%%%%%%%%%%%%%%%%%%%%%%%%%%%%%%%%%%%%%%%%%%%%%%%%%%%%%%%%%%%%%%%%%

\begin{table}[ht]
\centering
\begin{tabular}{rrrrrrrr}
  \hline
 & posterior & likelihood & prior & treeLikelihood & TreeHeight & YuleModel & birthRate \\ 
  \hline
1 & 10001 & 10001 & 10001 & 10001 & 9913 & 10001 & 9540 \\ 
   \hline
\end{tabular}
\caption{
  ESSes of the TMH-only alignment
  \richel{These are just example values}
} 
\label{tab:esses_tmh}
\end{table}

\begin{table}[ht]
\centering
\begin{tabular}{rrrrrrrr}
  \hline
 & posterior & likelihood & prior & treeLikelihood & TreeHeight & YuleModel & birthRate \\ 
  \hline
1 & 10001 & 10001 & 10001 & 10001 & 9913 & 10001 & 9540 \\ 
   \hline
\end{tabular}
\caption{
  ESSes of the non-TMH alignment
  \richel{These are just example values}
} 
\label{tab:esses_non_tmh}
\end{table}

\end{document}
