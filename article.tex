\documentclass{article}

% Bibliography
\usepackage{natbib}
\bibpunct{(}{)}{;}{a}{}{;}

% Use 'It was found that A is B (Name 1234)' style
\setcitestyle{authoryear,open={},close={}}

% Affiliations
\usepackage{authblk}
\title{pirouette: Determining the error BEAST2 makes in inferring a phylogeny}
\author[1]{Rich\`el J.C. Bilderbeek}
\author[1]{?Rampal S. Etienne}
\author[2]{Frans Bianchi}
\affil[1]{Groningen Institute for Evolutionary Life Sciences, University of Groningen, Groningen, The Netherlands}
\affil[2]{Frans' Institute, University of Groningen, Groningen, The Netherlands}

% Use double spacing
\usepackage{setspace}
\doublespacing

\usepackage{listings}
\usepackage{hyperref}
\usepackage{todonotes}
\usepackage{verbatim}
\usepackage{pgf}
\usepackage{bm}
\usepackage{multirow}
\usepackage{amsfonts}
\usepackage{array}
\usepackage{array}
\usepackage{booktabs}
\newcolumntype{C}[1]{>{\centering\arraybackslash}p{#1}}
\newcolumntype{L}[1]{>{\raggedright\arraybackslash}p{#1}}
\usepackage{longtable}

\usepackage{tkz-graph}
\usetikzlibrary{arrows,automata}
\usetikzlibrary{calc}
\usetikzlibrary{arrows.meta}

% sidewaysfigure
\usepackage{rotating}

% Style of listings
% From http://r.789695.n4.nabble.com/How-to-nicely-display-R-code-with-the-LaTeX-package-listings-tp4648110.html
\usepackage{fancyvrb} 
\definecolor{codegreen}{rgb}{0,0.6,0}
\definecolor{codegray}{rgb}{0.5,0.5,0.5}
\definecolor{codepurple}{rgb}{0.58,0,0.82}
\definecolor{backcolor}{rgb}{0.95,0.95,0.92}
\lstdefinestyle{mystyle}{
  language=R,% set programming language
  basicstyle=\ttfamily\small,% basic font style
  commentstyle=\color{gray},% comment style
  % numbers=left,% display line numbers on the left side
  numberstyle=\scriptsize,% use small line numbers
  numbersep=10pt,% space between line numbers and code
  tabsize=2,% sizes of tabs
  showstringspaces=false,% do not replace spaces in strings by a certain character
  captionpos=b,% positioning of the caption below
  breaklines=true,% automatic line breaking
  escapeinside={(*}{*)},% escaping to LaTeX
  fancyvrb=true,% verbatim code is typset by listings
  extendedchars=false,% prohibit extended chars (chars of codes 128--255)
  alsoletter={.<-},% becomes a letter
  alsoother={$},% becomes other
  otherkeywords={!=, ~, $, \&, \%/\%, \%*\%, \%\%, <-, <<-, /},% other keywords
  deletekeywords={c}% remove keywords 
}
\lstset{style=mystyle}

% Adds numbered lines
\usepackage{lineno}
\linenumbers

% Rename 'Abstract' to 'Summary 
\usepackage[english]{babel}
\addto{\captionsenglish}{\renewcommand{\abstractname}{Summary}}

%comments
\newcommand{\frans}[1]{\textcolor{blue}{\textbf{[FB: #1]}}}
\newcommand{\richel}[1]{\textcolor{orange}{\textbf{[RB: #1]}}}
\newcommand{\rampal}[1]{\textcolor{green}{\textbf{[RSE: #1]}}}


\begin{document}

\maketitle

\begin{abstract}

About a fourth of the human proteome is anchored by transmembrane helices (TMHs) to
lipid membranes. It, therefore, comes as no surprise that
bioinformatics analysis predicts an over-representation of TMHs among strong MHC
class I (MHC-I) binders. Published peptide elution studies confirm that TMHs are indeed
presented by MHC-I. This raises the question how membrane proteins are processed
for MHC-I (cross-)presentation, with current research focusing on soluble antigens. The
presentation of membrane-buried peptides is likely important in health and disease, as
TMHs are considerably conserved and their presentation might prevent escape muta-
tions by pathogens. Therefore, it could contribute to the disease correlations described
for many human leukocyte antigen haplotypes.

\end{abstract}

{\bf Keywords:} antigen presentation, antigen cross-presentation, membrane proteins, bioinformatics, adaptive
immunity, transmembrane domain, epitopes, T lymphocyte, MHC-2

%%%%%%%%%%%%%%%%%%%%%%%%%%%%%%%%%%%%%%%%%%%%%%%%%%%%%%%%%%%%%%%%%%%%%%%%%%%%%%%%%%%%%%
\section{Introduction}
%%%%%%%%%%%%%%%%%%%%%%%%%%%%%%%%%%%%%%%%%%%%%%%%%%%%%%%%%%%%%%%%%%%%%%%%%%%%%%%%%%%%%%

\cite{bianchi2017transmembrane}


%%%%%%%%%%%%%%%%%%%%%%%%%%%%%%%%%%%%%%%%%%%%%%%%%%%%%%%%%%%%%%%%%%%%%%%%%%%%%%%%%%%%%%
\section{Methods}
%%%%%%%%%%%%%%%%%%%%%%%%%%%%%%%%%%%%%%%%%%%%%%%%%%%%%%%%%%%%%%%%%%%%%%%%%%%%%%%%%%%%%%

There are thousands of MHC2 alleles known.
For this research, we picked the three alleles that are most abundant,
accoring to \richel{reference here}, 
which are DRB4*0101, 
HLA-DPA10103-DPB10402 
\richel{
  we decided for DPB1*0402, but there are multiple alleles matching.
  I just picked the first one of the alleles matching, which are:
  HLA-DPA10103-DPB10402,
  HLA-DPA10104-DPB10402,
  HLA-DPA10105-DPB10402,
  HLA-DPA10106-DPB10402,
  HLA-DPA10107-DPB10402,
  HLA-DPA10108-DPB10402,
  HLA-DPA10109-DPB10402,
  HLA-DPA10110-DPB10402,
  HLA-DPA10201-DPB10402,
  HLA-DPA10202-DPB10402,
  HLA-DPA10203-DPB10402,
  HLA-DPA10204-DPB10402,
  HLA-DPA10301-DPB10402,
  HLA-DPA10302-DPB10402,
  HLA-DPA10303-DPB10402,
  HLA-DPA10401-DPB10402
}
and DQA1*0501/DQB1*0301,
that are present in respectively 31, 41 and 59 percent 
\richel{fix numbers}
of the current human population.

%%%%%%%%%%%%%%%%%%%%%%%%%%%%%%%%%%%%%%%%%%%%%%%%%%%%%%%%%%%%%%%%%%%%%%%%%%%%%%%%%%%%%%
\section{Results}
%%%%%%%%%%%%%%%%%%%%%%%%%%%%%%%%%%%%%%%%%%%%%%%%%%%%%%%%%%%%%%%%%%%%%%%%%%%%%%%%%%%%%%

\begin{table}
\centering
  \begin{tabular}{| r | c | c | c | c | c | c |}
    \hline
    \textbf{Allele}       & \textbf{i} & \textbf{m} & \textbf{o} & \textbf{I} & \textbf{M} & \textbf{O} \\ 
    \hline
    DRB4*0101             & 3752 & 2397 & 62024 & 1061 & 854  & 11952 \\
    HLA-DPA10103-DPB10402 & 4335 & 2328 & 69040 & 478  & 923  & 4936 \\
    DQA1*0501/DQB1*0301   & 3321 & 1329 & 45461 & 1492 & 1922 & 28515 \\
    \hline 
  \end{tabular}
  \caption{
    \richel{These are preliminary, 6 percent of proteome}
    Location and binding of amino acids for the alleles.
    i: non-binding and inside, 
    m: non-binding and transmembrane, 
    o: non-binding and outside, 
    I: binding and inside, 
    M: binding and transmembrane, 
    O: binding and outside
  }
  \label{tab:results_old}
\end{table}

\input{table_imoimo.latex}
% has label tab:results


%%%%%%%%%%%%%%%%%%%%%%%%%%%%%%%%%%%%%%%%%%%%%%%%%%%%%%%%%%%%%%%%%%%%%%%%%%%%%%%%%%%%%%
\section{Conclusion}
%%%%%%%%%%%%%%%%%%%%%%%%%%%%%%%%%%%%%%%%%%%%%%%%%%%%%%%%%%%%%%%%%%%%%%%%%%%%%%%%%%%%%%

%%%%%%%%%%%%%%%%%%%%%%%%%%%%%%%%%%%%%%%%%%%%%%%%%%%%%%%%%%%%%%%%%%%%%%%%%%%%%%%%%%%%%%
\section{Discussion}
%%%%%%%%%%%%%%%%%%%%%%%%%%%%%%%%%%%%%%%%%%%%%%%%%%%%%%%%%%%%%%%%%%%%%%%%%%%%%%%%%%%%%%



%%%%%%%%%%%%%%%%%%%%%%%%%%%%%%%%%%%%%%%%%%%%%%%%%%%%%%%%%%%%%%%%%%%%%%%%%%%%%%%%%%%%%%
\section{Acknowledgements}
%%%%%%%%%%%%%%%%%%%%%%%%%%%%%%%%%%%%%%%%%%%%%%%%%%%%%%%%%%%%%%%%%%%%%%%%%%%%%%%%%%%%%%

Thanks to Geert van den Bogaart for his help \richel{reword}.
We thank the Netherlands 
Organization for Scientific Research (NWO) for financial support 
through a VICI grant awarded to RSE.

%%%%%%%%%%%%%%%%%%%%%%%%%%%%%%%%%%%%%%%%%%%%%%%%%%%%%%%%%%%%%%%%%%%%%%%%%%%%%%%%%%%%%%
\section{Data Accessibility}
%%%%%%%%%%%%%%%%%%%%%%%%%%%%%%%%%%%%%%%%%%%%%%%%%%%%%%%%%%%%%%%%%%%%%%%%%%%%%%%%%%%%%%

All code is archived at \url{http://github.com/richelbilderbeek/someplace},
with DOI \url{https://doi.org/12.3456/zenodo.1234567}.

%%%%%%%%%%%%%%%%%%%%%%%%%%%%%%%%%%%%%%%%%%%%%%%%%%%%%%%%%%%%%%%%%%%%%%%%%%%%%%%%%%%%%%
\section{Authors' contributions}
%%%%%%%%%%%%%%%%%%%%%%%%%%%%%%%%%%%%%%%%%%%%%%%%%%%%%%%%%%%%%%%%%%%%%%%%%%%%%%%%%%%%%%

RJCB, FB and RSE conceived the idea for this research. 
RJCB wrote the code.
FB supplied the DNA alignment.
RJCB, FB and RSE contributed to writing the article.

%%%%%%%%%%%%%%%%%%%%%%%%%%%%%%%%%%%%%%%%%%%%%%%%%%%%%%%%%%%%%%%%%%%%%%%%%%%%%%%%%%%%%%
% Bibliography
%%%%%%%%%%%%%%%%%%%%%%%%%%%%%%%%%%%%%%%%%%%%%%%%%%%%%%%%%%%%%%%%%%%%%%%%%%%%%%%%%%%%%%
% MEE style
\bibliographystyle{mee}
\bibliography{article}
%%%%%%%%%%%%%%%%%%%%%%%%%%%%%%%%%%%%%%%%%%%%%%%%%%%%%%%%%%%%%%%%%%%%%%%%%%%%%%%%%%%%%%

\appendix

%%%%%%%%%%%%%%%%%%%%%%%%%%%%%%%%%%%%%%%%%%%%%%%%%%%%%%%%%%%%%%%%%%%%%%%%%%%%%%%%%%%%%%
\section{Appendix section 1}
%%%%%%%%%%%%%%%%%%%%%%%%%%%%%%%%%%%%%%%%%%%%%%%%%%%%%%%%%%%%%%%%%%%%%%%%%%%%%%%%%%%%%%


\end{document}
