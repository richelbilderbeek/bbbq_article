\begin{abstract}

Cytolytic T cell responses are predicted to be biased towards membrane proteins. 
Because the peptide-binding grooves of most haplotypes 
of histocompatibility complex class I (MHC-I) are relatively hydrophobic, 
peptide fragments derived from transmembrane helices (TMHs) 
are predicted to be presented more often than expected based on their abundance.
However, the physiological reason of why membrane proteins might be 
over-presented is unclear
In this study, we show the TMHs are evolutionarily more conserved, 
as relatively less single nucleotide polymorphisms (SNPs) 
are present in TMH-coding chromosomal regions 
compared to regions coding for extracellular and cytosolic protein regions. 
Moreover, we show that the over-presentation of TMH-derived peptides is 
likely general, as it is predicted for diverse microbial pathogens 
and for both MHC-I and MHC-II.
Thus, our findings suggest that T cells might respond 
more to membrane proteins, because these are evolutionary more conserved.
We speculate that TMHs might be less prone of the escape mutations 
that enable pathogens to evade T cell responses.

\end{abstract}

{\bf Keywords:} antigen presentation, membrane proteins, bioinformatics, 
adaptive immunity, transmembrane domain, transmembrane helix, 
epitopes, T lymphocyte, MHC-I, MHC-II, evolutionary conservation

%%%%%%%%%%%%%%%%%%%%%%%%%%%%%%%%%%%%%%%%%%%%%%%%%%%%%%%%%%%%%%%%%%%%%%%%%%%%%%%%
\section{Introduction}
%%%%%%%%%%%%%%%%%%%%%%%%%%%%%%%%%%%%%%%%%%%%%%%%%%%%%%%%%%%%%%%%%%%%%%%%%%%%%%%%

% \paragraph{Immune response}

Our immune system fights microbial pathogens on a daily basis,
such as fungi, bacteria or viruses.
The acquired immune response
needs time to develop its specialized and more effective
combat forces and learn to recognize the pathogens 
by use of the Major Histocompatibility Complex (MHC).

% \paragraph{Immune response by MHC-I}

Antigen-presenting cells such as dendritic cells
use the MHC class I molecules to 
present randomly sampled peptides fragments,
as created by the immunoproteasome (\cite{chapiro2006destructive}).
If a cell gets infected by an intracellular pathogen, also the virus' peptides
will also be presented at the cell surface. The foreign viral antigen is detected 
by cytotoxic T lymphocytes, which will kill the infected cell.

% \paragraph{Immune response by MHC-II, contrast with APC}

All pathogens can be detected by the foreign proteins on 
their (bacterial or fungal) cell walls or (viral) envelope.
For bacteria and fungi, this is the main mode of their detection,
as these do not infect a cell with their (foreign) DNA.
T and B cells express MHC-II to detect foreign peptide fragments.
An immune response is started when MHC-II detects a pathogen.

% \paragraph{Classification of HLA}

Any human's immune system detects only a fraction of all possible
peptide fragments.
For humans, the MHC proteins are encoded by the
HLA (Human Leukocyte Antigens) genes.
There are three genes encoding for MHC-I, which are HLA-A, HLA-B and HLA-C,
for MHC-II there are three major genes, which are HLA-DR, HLA-DQ, HLA-DP.
Each MHC complex can only bind a subset of all possible peptides.
For example, HLA-A and HLA-B have no overlap in which
peptides they bind (\cite{lund2004definition}).
The HLA region of humans is highly polymorphic, with hundreds
to thousands of alleles for a gene (\cite{marsh2010nomenclature}).

% \paragraph{HLAs increase detection range}

Because of these multiple and highly polymorphic MHC genes,
the number of pathogenic peptides that can be detected is increased,
as a wide variety of MHCs will be expressed,
each presenting their own subset of epitopes.

% \paragraph{Epitope prediction}

It is helpful to be able to predict which peptides are immunogenic,
in, among others, vaccine development. 
Already for a decade, synthesized peptide fragments are used
to experimentally determine which peptides
are immunogenic.
This approach, however, is tedious and costly.
Due to this, software was written that allows for silico 
predictions, which are reliable in 
practice (\cite{larsen2010identification,schellens2008unanticipated,tang2011genome}).
 
% \paragraph{TMHs}

As the immunoproteasome is a cytosolic protein complex, on expects it
to cleave only cytosolic proteins. However, it has recently been
discovered that transmembrane protein residues are also presented
by MHC-I complexes (\cite{bianchi2017}). This discovery is based
on both in vitro elution studies, as well as on in silico predictions,
with a focus on transmembrane helices. 
The intracellular pathway, in which transmembrane residues are loaded onto
an MHC, however, is yet unknown.

Transmembrane helices (TMHs) 
are structurally conserved and hydrophobic 
regions of a protein that span a (also hydrophobic) 
cell membrane by an alpha helix of usually 23 amino acids.
Together with the internals of soluble proteins,
peptides derived from TMHs are the most hydrophobic
of a proteome. In both cases, the hydrophobic regions
help shape the protein in its correct 3D configuration.
TMHs are common structures in the proteins of
humans and pathogens. For example, 
15-39\% of the proteins in humans
is predicted to be a membrane protein (\cite{ahram2006estimation}).
There are multiple tools to do in silico predictions 
to determine the transmembrane regions of a protein 
(\cite{krogh2001predicting,bianchi2017,kall2004combined,arai2004conpred,jones2007improving,klammer2009metatm,wang2019efficient}).

% \paragraph{MHC-I presents TMH-derived epitopes in humans more often}

For MHC-I, it was found that predicted epitopes derived 
from human transmembrane helices (TMHs)
are over-presented by all 5 HLA-A and 
most of 8 HLA-B super-types (\cite{bianchi2017}),
due the high hydrophobicity as is required for a TMH to span
the (hydrophobic) cell membrane, combined with the fact that the MHC
binding cleft is hydrophobic as well.

It is unknown if the presentation of hydrophobic/TMH residues
is evolutionary selected for.
One reason to expect a reduced variability (and hence 
evolutionary conservation) in TMHs, is
that these are restricted in their evolution
by the functional requirement to span a lipid bilayer.
Due to this, the TMHs that pathogens possess, 
have a lower chance to develop an escape mutation,
as many mutations will result in a dysfunctional TMH.

% \paragraph{Does MHC-I present TMH-derived epitopes from pathogens as often?}

It is important that MHC-I presents both the peptides of the 
(healthy) cell, as well as possible pathogen-derived peptides when the
cell is infected. As described above, MHC-I presents TMH-derived epitopes 
from the human host more often. It is unknown if MHC-I has the same
dedication to present epitopes that stem from TMHs derived from
proteins produced by pathogen, either would the pathogen
be a virus or 
a bacterium.

% \paragraph{MHC-II is expected to present TMHs}

If presentation of TMHs on MHC-I would bring an evolutionary advantage 
in the recognition of pathogens by the immune system, 
it would follow that this is equally important for MHC-II, 
especially as the help of CD4+ T cells is needed for a long lasting CD8+ T cell 
response (\cite{novy2007cd4}). 
The mechanism to detect foreign TMHs by MHC-II would be unknown, 
similar to the discovery that MHC-I presents TMHs.

% \paragraph{Does MHC-II present TMH-derived epitopes from pathogens as often?}

It is unknown if MHC-II, like MHC-I, 
presents TMH-derived epitopes as often, either 
in humans,
bacteria or viruses.

% \paragraph{Selection undetectable in whole proteome}

In general, on would hope that evolutionary selection results in
an immune system that as most attentive for loci that are
essential for a virus, as these will be most conserved.
In SARS-CoV-2, for example, there is preliminary evidence that the strongest
selection pressure is upon residues that changes its 
virulence (\cite{velazquez2020positive}).
These loci, however, only account for a small part of a pathogen's proteome.
Additionally, these essential parts differ widely between pathogens.
Because of this scarcity and variance in targets, 
one can imagine that the human immune system 
is not tailored to detect these sites, 
as hinted by upon by the aforementioned influenza study.

% \paragraph{Selection may be detectable in TMHs}

TMHs, on the other hand, also have their function constraints, 
yet can occur multiple time a pathogen's proteome.
One can safely assume a pathogen's proteome contains multiple TMHs.
Therefore, it may be beneficial for the host
if its immune system would be more attentive towards TMHs.
And maybe this has already happened: MHC-I already detects hydrophobic
peptides. This feature, however, may also be caused by selection
to detect hydrophobic regions in the soluble proteins of pathogens.
It is unknown, when focusing on TMHs only, if a signal of selection
can be detected.

% \paragraph{Use of protein data in phylogenetics}

When using DNA sequences, one can use a skewed rate
in non- versus synonymous mutation
to detect the signature of selection (\cite{murrell2015gene}).
Using AA sequences, however, has its advantages,
as its is closer to the actual phenotype
selection acts upon: DNA may never be transcribed to RNA,
or its RNA may never be translated (\cite{diz2012proteomics}).
When using a proteome in phylogenetic research,
we know that the majority of proteins are selected to just
maintain their function most of the time, where
is the time spans there is selection, only a few AAs
can actually increase the 'fitness' of the
protein (\cite{anisimova2009investigating}).
There, when generalizing the dynamics of mostly purifying selection (to maintain
a protein's function) and a short duration of positive selection,
those genes that are selected cannot be detected (\cite{yang2000statistical}).

%%%%%%%%%%%%%%%%%%%%%%%%%%%%%%%%%%%%%%%%%%%%%%%%%%%%%%%%%%%%%%%%%%%%%%%%%%%%%%%%
\section{Methods}
%%%%%%%%%%%%%%%%%%%%%%%%%%%%%%%%%%%%%%%%%%%%%%%%%%%%%%%%%%%%%%%%%%%%%%%%%%%%%%%%

% \paragraph{Data sets for TMH epitopes}

To determine the percentages of epitopes overlapping
with TMHs, we use reference proteomes of 
a human, viral and bacterial nature.
An overview of Uniprot IDs is shown in table \ref{tab:uniprot_ids}.
Instead of using the full human reference proteome, we
use the representative human reference proteome 
as used previously (\cite{bianchi2017}).


%%%%%%%%%%%%%%%%%%%%%%%%%%%%%%%%%%%%%%%%%%%%%%%%%%%%%%%%%%%%%%%%%%%%%%%%%%%%%%%%
\subsection{Elution studies}\label{subsec:elution_studies}
%%%%%%%%%%%%%%%%%%%%%%%%%%%%%%%%%%%%%%%%%%%%%%%%%%%%%%%%%%%%%%%%%%%%%%%%%%%%%%%%

To determine if epitopes derived from TMHs are presented at all,
we started from epitopes sequenced from elution studies
for MHC-I (\cite{schellens2015comprehensive}) 
and MHC-II (\cite{bergseng2015different}).
For each epitope, the human representative reference proteome
was searched for its origin.
We kept only the epitopes of which were found exactly once
on the proteome.
We predicted the topology of the proteome
and counted how often the epitopes overlapped
with a predicted TMH.
In this analysis, both TMHMM and 
PureseqTM (see section 'Prediction software used') 
were used to predict the topology.
The full analysis can be found
at \url{https://github.com/richelbilderbeek/bbbq_article_issue_157}.

%%%%%%%%%%%%%%%%%%%%%%%%%%%%%%%%%%%%%%%%%%%%%%%%%%%%%%%%%%%%%%%%%%%%%%%%%%%%%%%%
\subsection{Measuring TMH epitopes}
%%%%%%%%%%%%%%%%%%%%%%%%%%%%%%%%%%%%%%%%%%%%%%%%%%%%%%%%%%%%%%%%%%%%%%%%%%%%%%%%

To determine the percentages of MHC-I and MHC-II epitopes overlap
with TMHs, we used mostly the same analysis as described in \cite{bianchi2017}.
To summarize: from a proteome all possible 9-mers are derived. For each
of these peptides, it is measured if it has been part of a 
TMH (at least overlapping 1 residue), 
as well as if the peptide is a binder to an MHC-I haplotype.
For MHC-II, 14-mers are used, as these are the most frequently occurring
epitope length (\cite{bergseng2015different}).

This study differs in some aspects, described below in more detail,
which is the definition of what a binder is,
the addition of using MHC-II haplotypes, the software used,
and the significance level to determine if TMH-derived peptides are as 
likely to be a binder as expected by chance.
These deviations are either a refinement of the previous method or
a pragmatic choice made due to the extension of the original experiment.
Additionally, instead of only using a human proteome, in this study
also a viral and bacterial proteome are analyzed.

The definition of a binder differs from \cite{bianchi2017}:
in this study a peptide is called a binder if, within a haplotype, 
any of its 9-mer peptides have an IC50 value in the lowest 2\% of 
the epitopes within a 
\emph{proteome} (see tables \ref{tab:tmh_binders_mhc1} and \ref{tab:tmh_binders_mhc2}
for values), whereas the original study defined
a binder as having an IC50 in the lowest 2\% 
of the peptides within a \emph{protein}.
% See https://github.com/richelbilderbeek/bianchi_et_al_2017/blob/72e6755a31d400158368509fd80a41e984677ab1/predict-binders.R#L17

The 13 MHC-I haplotypes used in this study are the same as 
the ones in \cite{bianchi2017}.
The MHC-II haplotypes used additionally are selected 
to occur with a phenotypic frequency of at least 14\% in
the human population (\cite{greenbaum2011functional}),
resulting in 21 haplotypes.
When using an MHC-II haplotype, instead of using 9-mers, 14-mers are
used, as these are the most common MHC-II epitope size,
as found in the elution study (\cite{bergseng2015different}).

%%%%%%%%%%%%%%%%%%%%%%%%%%%%%%%%%%%%%%%%%%%%%%%%%%%%%%%%%%%%%%%%%%%%%%%%%%%%%%%%
\subsubsection{Evolutionary conservation of TMHs}
%%%%%%%%%%%%%%%%%%%%%%%%%%%%%%%%%%%%%%%%%%%%%%%%%%%%%%%%%%%%%%%%%%%%%%%%%%%%%%%%

% \paragraph{Introduction}

To detect the evolutionary conservation of TMHs, 
we collect human mutations and 
tally their predicted location.
To be more precise, we collected single nucleotide
polymorphisms (SNPs) within the human population
that resulted in the substitution of one amino acid.
As we know the sequence of the protein a SNP acts upon,
we predict that protein's topology and tally if
the SNP occurred in a TMH or not, 
after which we use statistics to determine if there are
more, less or an equal amount of SNPs in TMHs
as a measure of evolutionary conservation.
This workflow is discussed in more detail below and also
depicted in figure \ref{fig:snp_workflow}).

% \paragraph{Data}

As a data source, multiple
NCBI (\url{https://www.ncbi.nlm.nih.gov/}) databases are used,
which are 'gene', to find the gene names of membrane proteins, 
'dbSNP' (\cite{sherry2001dbsnp}) for SNPs associated with those genes
and 'protein', to obtain the sequence of proteins that SNPs act upon.
The 'dbSNP' contains 650 million 
catalogued non-redundant humane variations (called RefSNPs,
\url{https://www.ncbi.nlm.nih.gov/snp/docs/RefSNP_about/}).
It is estimated that approximately ten percent (ranging from 2 to 36 percent) 
are artifacts (\cite{carlson2003additional, cutler2001high, gabriel2002structure, mitchell2004discrepancies, musumeci2010single, reich2003quality}),
yet if this has been corrected for (if possible) in the last decade is unknown.

% \paragraph{Data retrieval}

To retrieve the data from these databases the
\verb;rentrez; R package (\cite{rentrez}) is used
to calls the NCBI website's API. To provide for a 
stable user experience for all users, 
this API limits the user to 3 calls per second.
Additionally, the API splits the result of a bigger
query into multiple pages, each of which needs one API call.
We wrote the \verb;sprentrez; package (\cite{sprentrez}) to provide for 
bigger queries of multiple (and delayed) API calls.

% \paragraph{Pipeline}

The first query is a call to the 'gene' database for the 
term 'membrane protein' (in all fields) for the organism Homo sapiens.
This results in 1129 gene IDs (note that this number varies through time).
The next query is a call to the 'gene' database 
to obtain the gene names from the gene IDs.
Per gene name, the 'dbSNP' NCBI database is queried for 
variations associated with the gene name. 
The number of variations
was limited to the first 250 variations per gene,
resulting in 282250 variations. A variation need not always be a SNP,
as dbSNP also catalogs other DNA alterations, such as, among others, insertions,
deletions and tandem repeats.
We select only the variations that result in a SNP for
a single amino acid substitution.
Per SNP, the 'protein' NCBI database was queried for the
protein sequence.
Of each protein sequence, the protein topology is determined 
using PureseqTM.
From the topology and the known location of the SNP, 
we score the location (i.e. transmembrane or soluble protein region) 
where the change occurred.

%%%%%%%%%%%%%%%%%%%%%%%%%%%%%%%%%%%%%%%%%%%%%%%%%%%%%%%%%%%%%%%%%%%%%%%%%%%%%%%%
\subsection{Prediction software used}
\label{subsec:prediction_software_used}
%%%%%%%%%%%%%%%%%%%%%%%%%%%%%%%%%%%%%%%%%%%%%%%%%%%%%%%%%%%%%%%%%%%%%%%%%%%%%%%%

\begin{table}[]
  \begin{tabular}{llll}
    Goal & Tool & Reference \\ 
    \hline
    Predict topology                  & TMHMM                     & \cite{krogh2001predicting} \\
    Predict topology                  & PureseqTM                 & \cite{wang2019efficient} \\
    Predict epitopes MHC-I            & \verb;epitope-prediction; & \cite{bianchi2017} \\
    Predict epitopes MHC-II           & NetMHCIIpan               & \cite{nielsen2008quantitative,karosiene2013netmhciipan} \\
    Call TMHMM from R                 & tmhmm                     & \cite{tmhmm} \\
    Call PureseqTM from R             & pureseqtmr                & \cite{pureseqtmr} \\
    Call NetMHCIIpan from R           & netmhc2pan                & \cite{netmhc2pan} \\
    Combine all                       & bbbq                      & \cite{bbbq}
  \end{tabular}
  \caption{
    Overview of all software used in this research.
    'LUT' is an abbreviation for 'Look-up table'. 
  }
  \label{table:software_used}
\end{table}

For this research, the scientific literature was explored 
to find the most recent free and open source (FOSS) prediction software.
This was done by searching for papers that (1) reference older
prediction software, and (2) present a novel method to make predictions.
As a starting point, a review paper was used.

% \paragraph{TMH prediction}

There are multiple computational tools developed to predict which
parts of membrane proteins are TMH.
In 2001, multiple tools to do so have been compared (\cite{moller2001evaluation}),
of which TMHMM (\cite{krogh2001predicting}) turned out to be the best, 
as is used in the previous study (\cite{bianchi2017}).
TMHMM has a restrictive software license and is nearly two
decades old.
Due to this, also PureseqTM (\cite{wang2019efficient}),
is used in this study, which has been recently developed
and has a free software license.

% \paragraph{MHC-I epitope prediction}

For MHC-I, there are multiple computational tools developed 
to predict epitopes. 
According to \cite{lundegaard2011prediction}, at that time,
NetMHCcons (\cite{karosiene2012netmhccons}) gave the best predictions.
A tool developed later is \verb;epitope-prediction; (\cite{bianchi2017}),
which uses a stabilized matrix method (\cite{kim2009derivation}).
The most recent and promising FOSS tool available now appears
to be MHCnuggets (\cite{shao2020high}), which can do both MHC-I 
and MHC-II prediction.

% \paragraph{MHC-II epitope prediction}

Also for MHC-II, there are multiple computational tools developed 
to predict epitopes,
such as using a trained neural network (\cite{nielsen2003reliable})
or a Gibbs sampling approach (\cite{nielsen2004improved}).
According to \cite{lundegaard2011prediction}, in 2011,
from a set of multiple tools, 
NetMHCIIpan (\cite{nielsen2008quantitative,karosiene2013netmhciipan})
gave rise to the most accurate predictions.
Later tools are NetTepi (\cite{trolle2014nettepi}),
a genetic-algorithm-based ensemble method (\cite{zhang2015accurate})
and (again) the very recent MHCnuggets (\cite{shao2020high}).

%%%%%%%%%%%%%%%%%%%%%%%%%%%%%%%%%%%%%%%%%%%%%%%%%%%%%%%%%%%%%%%%%%%%%%%%%%%%%%%%
\subsection{Prediction software written}
%%%%%%%%%%%%%%%%%%%%%%%%%%%%%%%%%%%%%%%%%%%%%%%%%%%%%%%%%%%%%%%%%%%%%%%%%%%%%%%%

The R programming language is used for the complete 
experiment, including the analysis.
The complete experiment is bundled in the 'bbbq' R package,
which is dependent on 'tmhmm', 'pureseqtmr', 
'epitope-prediction' and 'mhcnuggetsr'
as described below.

% \paragraph{tmhmm}

The R package 'tmhmm' was developed to do the same topology
predictions as the earlier study (that used 'TMHMM'), yet in a reproducible way.
'TMHMM' has a restrictive software license and allows a user
to download a pre-compiled executable after confirmation he/she
is in academia. The R package respects this restriction
and allows the user to install and use TMHMM from within R,
as done in this study.
'tmhmm' has been submitted to and is accepted by CRAN.

% \paragraph{pureseqtmr}

To be able to call, from R, the TMH prediction software 'PureseqTM',
which is written in C, the package 'pureseqtmr' has been developed. 
'pureseqtmr' allows to install 'PureseqTM' and use most of its features.
Excluded are the features that are used by the 'PureseqTM' 
developers to verify the correctness of their work.
'pureseqtmr' has been submitted to and is accepted by CRAN.

% \paragraph{mhcnuggetsr}

MHCnuggets is a free and open-source Python package to predict 
epitope affinity for many MHC-I and MHC-II variants.
The R package 'mhcnuggetsr' allows one to install and use MHCnuggets
from within R.
Also 'mhcnuggetsr' has been submitted to and is accepted by CRAN.

% \paragraph{bbbq}

To reproduce the full experiment as done in this paper,
the functions needed are bundled in the 'bbbq' R package.
This package is too specific to be submitted to CRAN.

%%%%%%%%%%%%%%%%%%%%%%%%%%%%%%%%%%%%%%%%%%%%%%%%%%%%%%%%%%%%%%%%%%%%%%%%%%%%%%%%
\section{Results}
%%%%%%%%%%%%%%%%%%%%%%%%%%%%%%%%%%%%%%%%%%%%%%%%%%%%%%%%%%%%%%%%%%%%%%%%%%%%%%%%

%%%%%%%%%%%%%%%%%%%%%%%%%%%%%%%%%%%%%%%%%%%%%%%%%%%%%%%%%%%%%%%%%%%%%%%%%%%%%%%%
\subsection{Elution studies}
%%%%%%%%%%%%%%%%%%%%%%%%%%%%%%%%%%%%%%%%%%%%%%%%%%%%%%%%%%%%%%%%%%%%%%%%%%%%%%%%

% tab:elution
% latex table generated in R 4.1.1 by xtable 1.8-4 package
% Fri Oct 29 12:40:25 2021
\begin{table}[ht]
\centering
\begin{tabular}{llll}
  \hline
MHC class & Tool & Dataset & n \\ 
  \hline
I & PureseqTM & schellens & 1.38\% (109/7897) \\ 
  I & PureseqTM & iedb & 6.81\% (43/631) \\ 
  I & TMHMM & schellens & 1.43\% (113/7897) \\ 
  I & TMHMM & iedb & 7.13\% (45/631) \\ 
  II & PureseqTM & bergseng & 3.92\% (498/12712) \\ 
  II & PureseqTM & iedb & 0.29\% (4/1364) \\ 
  II & TMHMM & bergseng & 3.96\% (504/12712) \\ 
  II & TMHMM & iedb & 1.39\% (19/1364) \\ 
   \hline
\end{tabular}
\caption{Percentage of epitopes derived from a TMH found in the two elution studies, for the two different kind of topology prediction tools. The values between braces show the the number of epitopes that were predicted to overlapping with a TMH per all epitopes that could be uniquely mapped to the representative human reference proteome.} 
\label{tab:elution}
\end{table}


Table \ref{tab:elution} shows the percentage of epitopes derived from a TMH
found in the two elution studies, for the two different kind of topology
prediction tools. Whatever the tool or MHC class, at least 100 epitopes
can be uniquely map to human representative reference proteome.

%%%%%%%%%%%%%%%%%%%%%%%%%%%%%%%%%%%%%%%%%%%%%%%%%%%%%%%%%%%%%%%%%%%%%%%%%%%%%%%%
\subsection{MHC-I}
%%%%%%%%%%%%%%%%%%%%%%%%%%%%%%%%%%%%%%%%%%%%%%%%%%%%%%%%%%%%%%%%%%%%%%%%%%%%%%%%

Figure \ref{fig:1} shows the percentages of MHC-I epitopes overlapping 
with TMHs for our human, viral and bacterial proteome.
See the supplementary materials (table \ref{tab:tmh_binders_mhc1}) 
for the exact TMH and epitope counts.

\begin{figure}[!htbp]
  \includegraphics[height=0.9\textheight]{bbbq_1_smart_results/fig_f_tmh_mhc1_2_grid.png}
  \caption{
    Percentage of MHC-I epitopes overlapping with TMHs
    for a human, viral and bacterial proteome.
    The horizontal lines indicate the percentage as expected by chance.
    See table \ref{tab:tmh_binders_mhc1} for the exact TMH and epitope counts.
  }
  \label{fig:1}
\end{figure}

%%%%%%%%%%%%%%%%%%%%%%%%%%%%%%%%%%%%%%%%%%%%%%%%%%%%%%%%%%%%%%%%%%%%%%%%%%%%%%%%
\subsection{MHC-II}
%%%%%%%%%%%%%%%%%%%%%%%%%%%%%%%%%%%%%%%%%%%%%%%%%%%%%%%%%%%%%%%%%%%%%%%%%%%%%%%%

Figure \ref{fig:2} shows the percentages of MHC-II epitopes overlapping 
with TMHs for our human, viral and bacterial proteome.
See the supplementary materials (table \ref{tab:tmh_binders_mhc2}) 
for the TMH and epitope counts.

\begin{figure}[!htbp]
  \includegraphics[height=0.9\textheight]{bbbq_1_smart_results/fig_f_tmh_mhc2_2_grid.png}
  \caption{
    Percentage of MHC-II epitopes overlapping with TMHs
    for a human, viral and bacterial proteome.
    The horizontal lines indicate the percentage as expected by chance.
    See table \ref{tab:tmh_binders_mhc2} for the exact TMH and epitope counts.
    Note that for smaller proteomes a percentage of zero is likelier.
  }
  \label{fig:2}
\end{figure}

%%%%%%%%%%%%%%%%%%%%%%%%%%%%%%%%%%%%%%%%%%%%%%%%%%%%%%%%%%%%%%%%%%%%%%%%%%%%%%%%
\subsection{Evolutionary conservation}
%%%%%%%%%%%%%%%%%%%%%%%%%%%%%%%%%%%%%%%%%%%%%%%%%%%%%%%%%%%%%%%%%%%%%%%%%%%%%%%%

\begin{figure}[!htbp]
  \includegraphics[width=\textwidth]{ncbi_peregrine_results/fig_f_snps_found_and_expected.png}
  \caption{
    Percentage of SNPs found in TMHs.
    Each point resembles one protein, with a predicted percentage of
    TMH (x-axis) and an observed occurrence of SNPs being located
    within a TMH (y-axis).
    The dashed diagonal line shows the expected linear trend line
    if TMHs and soluble protein regions are equally conserved.
    The red line is a linear trend line that includes the membrane-associated
    proteins at the origin. 
    The blue line is a linear trend line that includes only the
    transmembrane proteins.
  }
  \label{fig:f_snps_found_and_expected}
\end{figure}

We obtained 1129 gene names associated with the key phrase 'membrane protein'.
These genes are linked to 5203 proteins, of which 2754
are predicted to be transmembrane proteins.
We obtained 4811 variations that resulted in an
amino acids substitution, of which 2568 were located 
in predicted transmembrane proteins.
Per protein, we predicted the percentage of TMH
and calculated the percentage of amino acids substitutions
occurring in transmembrane regions.
These points, one per protein, were plotted as depicted in figure
\ref{fig:f_snps_found_and_expected}.
We did a linear fit on the data, with and without
membrane-associated proteins, and added a 95\%
confidence interval.
Also, we added the line \inlineequation{y = x},
which indicates the linear fit that would be expected if
SNPs occur just as likely within transmembrane and soluble
protein domains. Already these trend lines
hint that variations occur more often in soluble 
than in transmembrane protein regions.

To test whether SNPs occur just as likely in soluble as
in transmembrane protein domains, we used a Poisson binomial
distribution. We first calculated the expected number of SNPs
found in TMHs, which is 61457 SNPs multiplied by 6.8\% 
of all proteins being TMH, resulting in the null hypothesis
of 4203 SNPs being in the TMH. 
Note that including or excluding
the membrane associated proteins 
has no effect on this statistical test, as the chance 
of a successful trial is zero (there is zero chance
to find a SNP in TMHs if there are none).
The actually number of SNPs found in TMHs
was 3889, which is lower than the null hypothesis. 
The chance to find, within TMHs, this amount or less SNPs 
equals $1.6201 \cdot 10^{-9}$.
Using a alpha value of $0.05$, as is common practice for an uninformed
expectation, we concluded that there are significantly less SNPs
in TMHs as expected by chance.
The effect size is that per 1 SNP found in soluble protein
domains, one finds 0.925 SNPs in TMHs.

\begin{figure}[!htbp]
  \includegraphics[width=\textwidth]{ncbi_peregrine_results/fig_f_snps_found_and_expected_per_n_tmhs.png}
  \caption{
    Percentage of SNPs found in TMHs,
    separated per number of TMHs.
    The dashed diagonal lines show the expected linear trend line
    if TMHs and soluble protein regions are equally conserved.
  }
  \label{fig:f_snps_found_and_expected_per_n_tmhs}
\end{figure}
)

We decided to split this analysis based on the number of TMHs
a protein has. We hypothesize that single-spanners (i.e. proteins
with one TMH predicted) are less conserved, when compared to multi-spanners,
as single spanners just need to span a membrane, where some multi-spanners
have enzymatically active sites embedded within their TMHs and thus
have an additional structural constraint.
As can be seen in figure \ref{fig:f_snps_found_and_expected_per_n_tmhs}, 
there is no clear pattern to be observed.

%%%%%%%%%%%%%%%%%%%%%%%%%%%%%%%%%%%%%%%%%%%%%%%%%%%%%%%%%%%%%%%%%%%%%%%%%%%%%%%%
\section{Conclusion}
%%%%%%%%%%%%%%%%%%%%%%%%%%%%%%%%%%%%%%%%%%%%%%%%%%%%%%%%%%%%%%%%%%%%%%%%%%%%%%%%

% \paragraph{General pattern in epitope presentation}

We found a general pattern in the
presentation of epitopes that overlap with a TMH.
For humans, a bacterial and viral pathogen,
as well as for MHC-I and MHC-II,
we found that these are presented more often 
than expected by chance.

% \paragraph{Interesting haplotype for SARS-CoV-2}

It is known that an individual's haplotype
influences the dynamics of a pathogen (see,
for example, \cite{eccleston2017host} for HIV).
The bioinformatics pipeline developed in this
study may be interesting to make in silico
predictions of a SARS-CoV-2 infection.
The most prominent example is the (MHC-I)
haplotype HLA-A*24:02,
which is predicted to present four times as much epitopes derived 
from SARS-CoV-2 TMHs than expected by chance.
This may help the immune system to detect a SARS-CoV-2 infection earlier,
and change the infection's dynamics.

%%%%%%%%%%%%%%%%%%%%%%%%%%%%%%%%%%%%%%%%%%%%%%%%%%%%%%%%%%%%%%%%%%%%%%%%%%%%%%%%
\section{Discussion}
%%%%%%%%%%%%%%%%%%%%%%%%%%%%%%%%%%%%%%%%%%%%%%%%%%%%%%%%%%%%%%%%%%%%%%%%%%%%%%%%

\subsection{Elution studies}

% \paragraph{False positives}

We assume that epitopes 
derived from TMHs are actually presented by MHC-I and MHC-II.
Our analysis (at subsection \ref{subsec:elution_studies}) 
finds that 1.4\% (for MHC-I) and 4.0% (for MHC-II) 
of the epitopes presented 
overlap with a TMH.
It is unsure if this a true finding or a false positive.
One source of noise would stem from imperfect predictions
by the topology prediction software used. 
We have counteracted this by using two different topology prediction
software and confirm these have similar results.
Two biological sources of noise
are defective ribosomal products (DRiPs) (\cite{yewdell1996defective}) 
and peptide fusion \cite{delong2016pathogenic},
two processes that both result in peptide fragments
that did not originate from genome directly.
Note however, that both elution studies have a
methodological bias caused by mass spectronomy,
as hydrophobic fragments are detected less 
by mass spectronomy.

\subsection{\% epitopes overlapping with TMHs}

% \paragraph{Removal of selenoproteins}

We have removed the 25 selenoproteins found in the human
reference proteome, as both prediction tools struggle with 
selenocysteine, the 21st amino acid incorporated in human proteins.
TMHMM gives a clear error message, forcing us to remove
these selenoproteins from the analysis.
How PureseqTM deals with selenocysteine is unclear,
as it gives no error, nor is it mentioned anywhere,
nor do the authors respond to emails. Due to this, we removed the
selenoproteins from our analysis.

% \paragraph{Topology prediction of TRDD1}

PureseqTM does not predict the topology
of proteins that have less than three amino acids. 
The TRDD1 ('T cell receptor delta diversity 1') protein,
however, is two peptides long. 
We corrected for this by predicting that this protein
is completely cytosolic.

% \paragraph{Bacteria have a different cell membrane}

In this experiment we predicted epitopes that overlap with 
TMHs from a human, bacterial and viral proteome,
would these proteins be expressed in a human host.
Bacteria, however have different cell membranes and cell walls, 
hence different structural requirements for a TMH.
Both topology prediction tools were trained to recognize
human TMHs, thus we cannot be sure that
the transmembrane regions predicted in bacterial proteins
are actually part of an integral membrane protein.
For the purpose of this study, we assume the 
error in topology predictions to be unbiased way towards topology.
In other words: that a bacterial TMH is incorrectly
predicted to be absent just as often as it is incorrectly
predicted to be present elsewhere.

\subsection{Evolutionary conservation}

% \paragraph{synonymous mutations}

In our evolutionary experiment, 
we removed variations that were synonymous mutations (i.e.
resulted in the same amino acid, from a different genetic code) 
from our analysis.
There is evidence, however, that these synonymous mutations
do have an effect and may even be evolutionary selected 
for (\cite{hunt2009silent}).
As the possible effect of synonymous mutations is ignored by our
topology prediction software, we do so as well.

% \paragraph{False positives in SNPs}

Regarding the evolutionary conservation of TMHs using SNPs,
again, it is estimated that approximately ten percent
of SNPs is a false positive that result from the methods to determine
a SNP. One example is that sequence variations are incorrectly
detected due to highly similar duplicated sequences \cite{musumeci2010single}.
We assume that these duplications occur as often in TMHs as in
regions around these, hence we expect this not to affect our results.

% \paragraph{Low number of SNPs}

The NCBI dbSNP database contains millions of SNPs.
As the NCBI API constrains its users to three calls per second,
we had to limited the extent of our analysis
by downloading SNPs to approximately 300 
hours (i.e. the length of a Christmas holiday).

% \paragraph{Selection of SNPs}

We selected to analyze SNPs with the lowest SNP ID first.
This means we have a bias for picking SNPs with
an earlier discovery date.
We expect this bias to be biologically irrelevant
and we find no clear evidence of a bias, as
shown in figure \ref{fig:n_tmhs_per_protein}.
Still, the best way to interpret this figure
is open to discussion.

% We concluded that the
% epitopes that MHC-I presents are [as/not as] likely 
% to be derived from TMH within either a human host and its bacterial pathogen.
% Because a bacterium does not infect a cell, thus its peptides
% will not be presented by MHC-I, this result is [unexpected/expected]
% 
% We aimed our evolutionary experiment at TMHs, because these can
% be predicted well from a protein structure,
% are common structures and are present in all pathogens. 
% We could have done the same experiment on beta-turn,
% as also these can be predicted well (\cite{petersen2010netturnp}),
% are common structures and are present in all pathogens.
% 
% The human immune system and human pathogen are in an evolutionary
% arms race: our immune systems is selected for the detection
% of pathogens, whereas pathogens are selected to avoid detection.
% From a pathogen's point of view, however, this struggle 
% is of only minor importance:
% in seasonal influenza, for example, the selection pressure
% exerted by the immune system was only limited (\cite{han2019individual}).

%%%%%%%%%%%%%%%%%%%%%%%%%%%%%%%%%%%%%%%%%%%%%%%%%%%%%%%%%%%%%%%%%%%%%%%%%%%%%%%%
\section{Acknowledgments}
%%%%%%%%%%%%%%%%%%%%%%%%%%%%%%%%%%%%%%%%%%%%%%%%%%%%%%%%%%%%%%%%%%%%%%%%%%%%%%%%

We thank the Center for Information Technology of the University 
of Groningen for its support and for providing access to the Peregrine 
high performance computing cluster. 
Additionally, we would like to thank Sci-Hub (\cite{himmelstein2018sci})
for allowing us to read paywalled articles while working from home.

%%%%%%%%%%%%%%%%%%%%%%%%%%%%%%%%%%%%%%%%%%%%%%%%%%%%%%%%%%%%%%%%%%%%%%%%%%%%%%%%
\section{Data Accessibility}
%%%%%%%%%%%%%%%%%%%%%%%%%%%%%%%%%%%%%%%%%%%%%%%%%%%%%%%%%%%%%%%%%%%%%%%%%%%%%%%%

All code is archived at \url{http://github.com/richelbilderbeek/someplace},
with DOI \url{https://doi.org/12.3456/zenodo.1234567}.

%%%%%%%%%%%%%%%%%%%%%%%%%%%%%%%%%%%%%%%%%%%%%%%%%%%%%%%%%%%%%%%%%%%%%%%%%%%%%%%%
\section{Authors' contributions}
%%%%%%%%%%%%%%%%%%%%%%%%%%%%%%%%%%%%%%%%%%%%%%%%%%%%%%%%%%%%%%%%%%%%%%%%%%%%%%%%

RJCB and FB conceived the idea for this research. 
RJCB wrote the code.
RJCB and FB wrote the article.

%%%%%%%%%%%%%%%%%%%%%%%%%%%%%%%%%%%%%%%%%%%%%%%%%%%%%%%%%%%%%%%%%%%%%%%%%%%%%%%%
% Bibliography
%%%%%%%%%%%%%%%%%%%%%%%%%%%%%%%%%%%%%%%%%%%%%%%%%%%%%%%%%%%%%%%%%%%%%%%%%%%%%%%%
% MEE style
\bibliographystyle{mee}
\bibliography{bbbq_article}
%%%%%%%%%%%%%%%%%%%%%%%%%%%%%%%%%%%%%%%%%%%%%%%%%%%%%%%%%%%%%%%%%%%%%%%%%%%%%%%%


%%%%%%%%%%%%%%%%%%%%%%%%%%%%%%%%%%%%%%%%%%%%%%%%%%%%%%%%%%%%%%%%%%%%%%%%%%%%%%%%
\appendix
\section{Supplementary materials}
%%%%%%%%%%%%%%%%%%%%%%%%%%%%%%%%%%%%%%%%%%%%%%%%%%%%%%%%%%%%%%%%%%%%%%%%%%%%%%%%

%%%%%%%%%%%%%%%%%%%%%%%%%%%%%%%%%%%%%%%%%%%%%%%%%%%%%%%%%%%%%%%%%%%%%%%%%%%%%%%%
\subsection{Differences with Bianchi et al., 2017}
%%%%%%%%%%%%%%%%%%%%%%%%%%%%%%%%%%%%%%%%%%%%%%%%%%%%%%%%%%%%%%%%%%%%%%%%%%%%%%%%

A part of this study does the same analysis as Bianchi et al., 2017.
Here we describe the deviations, which are about the use of different
software and the use of a different definition of what a binder is.

% \paragraph{TMH prediction software}

The earlier study uses TMHMM, which has a restricted software license.
Instead, we chose to use PureseqTM, which is free and open source.

% Percentage of spots and spots that overlap with a TMH
\input{bbbq_1_smart_results/table_f_tmh_2.latex}

% \paragraph{IC50 prediction software}

The earlier study uses \verb;epitope-prediction; a hand-crafted method, 
that has been trained on MHC-I haplotypes only,
which is used here again. For MHC-II IC50 predictions, the
'netmhc2pan' R package is used, which calls NetMHC2pan.
Both are packages are free and open-source.

% \paragraph{Definition of what a binder is}

The earlier study defines a peptide a binder (for a haplotype), 
if \emp{within the peptide} in which it is found, 
is within the peptides with the 2\% lowest IC50 values.
This can be seen at \url{https://github.com/richelbilderbeek/bianchi_et_al_2017/blob/master/predict-binders.R},
where the binders are written to file.

In this study, a peptide is defined a binder (for a haplotype), 
if \emp{within a proteome} in which it is found, 
is within the peptides with the 2\% lowest IC50 values.
Subsection \ref{subsec:ic50s_per_haplotype} shows the IC50 values
for a binder per haplotype.

% \paragraph{Unclear handing of sequences with U}

Within the human reference proteome, there are 30 sequences that have a
selenocysteine (with the symbol 'U').
See \url{https://github.com/richelbilderbeek/bbbq_article/issues/158}.

%%%%%%%%%%%%%%%%%%%%%%%%%%%%%%%%%%%%%%%%%%%%%%%%%%%%%%%%%%%%%%%%%%%%%%%%%%%%%%%%
\subsection{IC50s per haplotype}
\label{subsec:ic50s_per_haplotype}
%%%%%%%%%%%%%%%%%%%%%%%%%%%%%%%%%%%%%%%%%%%%%%%%%%%%%%%%%%%%%%%%%%%%%%%%%%%%%%%%

Per target proteome (i.e. human, SARS-CoV-2, Mycobacterium tuberculosis),
we collected all 9-mers (for MHC-I) and 14-mers (for MHC-II),
after removing the selenoproteins and proteins that are shorter
than the epitope length.
From these epitopes, per MHC haplotype,
we predicted the IC50 (in nM) using \verb;epitope-prediction; (for MHC-I)
and MHCnuggets (for MHC-II). 
Here, we show the IC50 value per haplotype that
is used to determine if a peptide binds to the haplotype's MHC
for MHC-I (see table \ref{tab:ic50_binders_mhc1}) and 
MHC-II (see table \ref{tab:ic50_binders_mhc2}).

% tab:ic50_binders_mhc1
\input{bbbq_1_smart_results/table_ic50_binders_mhc1_2.latex}

% tab:ic50_binders_mhc2
\input{bbbq_1_smart_results/table_ic50_binders_mhc2_2.latex}

%%%%%%%%%%%%%%%%%%%%%%%%%%%%%%%%%%%%%%%%%%%%%%%%%%%%%%%%%%%%%%%%%%%%%%%%%%%%%%%%
\subsection{MHC-I}
%%%%%%%%%%%%%%%%%%%%%%%%%%%%%%%%%%%%%%%%%%%%%%%%%%%%%%%%%%%%%%%%%%%%%%%%%%%%%%%%

\begin{figure}[!htbp]
  \includegraphics[width=\textwidth]{bbbq_1_smart_results/fig_f_tmh_mhc1_2_normalized.png}
  \caption{
    Normalized proportion of MHC-I epitopes overlapping with TMHs
    for human, viral and bacterial proteomes.
    Legend: covid = SARS-CoV-2,
    human = homo sapiens, myco = Mycobacterium tuberculosis
  }
  \label{fig:f_tmh_mhc1_normalized}
\end{figure}

% Label: tab:tmh_binders_mhc1
\input{bbbq_1_smart_results/table_tmh_binders_mhc1_2.latex}

%%%%%%%%%%%%%%%%%%%%%%%%%%%%%%%%%%%%%%%%%%%%%%%%%%%%%%%%%%%%%%%%%%%%%%%%%%%%%%%%
\subsection{MHC-II}
%%%%%%%%%%%%%%%%%%%%%%%%%%%%%%%%%%%%%%%%%%%%%%%%%%%%%%%%%%%%%%%%%%%%%%%%%%%%%%%%

\begin{figure}[!htbp]
  \includegraphics[width=\textwidth]{bbbq_1_smart_results/fig_f_tmh_mhc2_2_normalized.png}
  \caption{
    Normalized proportion of MHC-II epitopes overlapping with TMHs
    for human, viral and bacterial proteomes.
    Legend: covid = SARS-CoV-2,
    human = homo sapiens, myco = Mycobacterium tuberculosis
  }
  \label{fig:f_tmh_mhc2_normalized}
\end{figure}

% Label: tab:tmh_binders_mhc2
\input{bbbq_1_smart_results/table_tmh_binders_mhc2_2.latex}

%%%%%%%%%%%%%%%%%%%%%%%%%%%%%%%%%%%%%%%%%%%%%%%%%%%%%%%%%%%%%%%%%%%%%%%%%%%%%%%%
\subsection{Evolutionary conservation}
%%%%%%%%%%%%%%%%%%%%%%%%%%%%%%%%%%%%%%%%%%%%%%%%%%%%%%%%%%%%%%%%%%%%%%%%%%%%%%%%

\begin{figure}[!htbp]
  \includegraphics[width=\textwidth]{ncbi_peregrine_results/fig_n_proteins_per_gene_name.png}
  \caption{
    Histogram of the number of proteins found per gene name.
    Most often, a gene name is associated with one proteins. 
  }
  \label{fig:n_proteins_per_gene_name}
\end{figure}

\begin{figure}[!htbp]
  \includegraphics[width=\textwidth]{ncbi_peregrine_results/fig_n_tmhs_per_protein.png}
  \caption{
    Histogram of the number of TMHs predicted per protein,
    for the integral membrane proteins used.
  }
  \label{fig:n_tmhs_per_protein}
\end{figure}

\begin{figure}[!htbp]
  \includegraphics[width=\textwidth]{ncbi_peregrine_results/fig_snp_rel_pos.png}
  \caption{
    Distribution of the relative position of the SNPs used,
    where a relative position of zero denotes the first amino
    acid at the N-terminus, where a relative position of one
    indicates the last residue at the C-terminus.
  }
  \label{fig:snp_rel_pos}
\end{figure}

%
% Notes to self
%

%\begin{sidewaystable}
%  \centering
%  ... centered table here
%\end{sidewaystable}

% \scalebox{0.7}{
%  ... scaled table here
% }

%{\tiny
%  ... something with a tiny font
%}

% Instead of non-TMH use 'soluble protein region'
% or 'cytosolic or extracellular protein region'

