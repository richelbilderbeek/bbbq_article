\begin{abstract}

Cytolytic T cell responses are predicted to be biased towards membrane proteins. 
Because the peptide-binding grooves of most haplotypes 
of histocompatibility complex class I (MHC-I) are relatively hydrophobic, 
peptide fragments derived from transmembrane helices (TMHs) 
are predicted to be presented more often than expected,
based on their abundance in the proteome.
We show that the over-presentation of TMH-derived peptides is likely general, 
as it is predicted for diverse microbiota and viruses 
and for both MHC-I and MHC-II
However, the physiological reason of why membrane proteins might be 
over-presented is unclear
In this study, we show the TMHs are evolutionarily more conserved, 
as relatively less single nucleotide polymorphisms (SNPs) 
are present in TMH-coding chromosomal regions 
compared to regions coding for extracellular and cytosolic protein regions. 
Thus, our findings suggest that T cells might respond 
more to membrane proteins, because these are evolutionary more conserved.
We speculate that TMHs therefor might be less prone to escape mutations 
that enable pathogens to evade T cell responses.

\end{abstract}

{\bf Keywords:} antigen presentation, membrane proteins, bioinformatics, 
adaptive immunity, transmembrane domain, transmembrane helix, 
epitopes, T lymphocyte, MHC-I, MHC-II, evolutionary conservation

%%%%%%%%%%%%%%%%%%%%%%%%%%%%%%%%%%%%%%%%%%%%%%%%%%%%%%%%%%%%%%%%%%%%%%%%%%%%%%%%
\section{Introduction}
%%%%%%%%%%%%%%%%%%%%%%%%%%%%%%%%%%%%%%%%%%%%%%%%%%%%%%%%%%%%%%%%%%%%%%%%%%%%%%%%

% \paragraph{Immune response}

Our immune system fights diseases and infections from pathogens, 
such as fungi, bacteria or viruses. 
An important part of the acquired immune response, 
that develops specialized and more effective recognition of pathogens, 
are T cells which recognize peptides derived from 
exogenous antigens presented on Major Histocompatibility Complexes (MHC). 

% \paragraph{Classification of HLA}

Any individual's immune system detects only a fraction of all possible
peptide fragments.
For humans, the MHC proteins are encoded by the
HLA (Human Leukocyte Antigens) genes.
There are three genes encoding for MHC-I, which are HLA-A, HLA-B and HLA-C,
for MHC-II there are three major genes, which are HLA-DR, HLA-DQ, HLA-DP.
Each MHC complex can only bind a subset of all possible peptides.
For example, HLA-A and HLA-B have no overlap in which
peptides they bind (\cite{lund2004definition})
[FB: is that really so?]
[RB: yes, according to the removed reference to Lund et al., 2004, 'Definition of supertypes for {HLA} molecules using clustering of specificity matrices'] 
The HLA region of humans is highly polymorphic, with hundreds 
to thousands of different alleles, 
and each different HLA gene is called 
an MHC haplotype (\cite{marsh2010nomenclature}).

% \paragraph{HLAs increase detection range}

Because of these multiple and highly polymorphic MHC genes,
the number of pathogenic peptides that can be detected is increased,
as a wide variety of MHCs: 
6 MHC-I and 6-8 MHC-II molecules are expressed per individual
each presenting their own subset of epitopes.
It is believed that this improves immunity at the population level, 
as mutations in a protein that disrupt a particular MHC presentation, 
so-called escape mutations, 
will not affect MHC presentation for all haplotypes (\cite{sommer2005importance}).

% \paragraph{Epitope prediction}

Many studies are aimed at determining which peptides are presented in MHC 
and will result in an immune response, 
as this will for instance aid the design of vaccines. 
These studies have led to the development 
of very reliable prediction algorithms 
that allows for in silico predictions 
of the binding affinities of peptides (\cite{larsen2010identification,schellens2008unanticipated,tang2011genome}).
 
% \paragraph{TMHs}

Using these prediction algorithms, 
we recently showed that peptides derived 
from transmembrane helices (TMHs) 
are predicted to be more frequently presented by MHC-I 
than expected based on their abundance (\cite{bianchi2017}). 
Moreover, we showed that some well-known immunodominant peptides stem from TMHs. 

This over-presentation is attributed to the fact 
that the peptide-binding groove of most MHC-I haplotypes 
is relatively hydrophobic, 
and therefore hydrophobic peptides have a higher affinity 
than hydrophilic peptides. 

Transmembrane helices (TMHs) are hydrophobic 
as they need to span the hydrophobic lipid bilayer of cellular membranes.
They consist of an alpha helix on average 23 amino acids in length. 
TMHs are common structures in the proteins of humans and pathogens. 
Different TMH prediction tools estimate
that 15-39\% of the proteins in humans 
contain at least one TMH (\cite{ahram2006estimation}).
TMHs can also be predicted from a protein sequence 
with high accuracy by bioinformatics approaches (\cite{krogh2001predicting,bianchi2017,kall2004combined,arai2004conpred,jones2007improving,klammer2009metatm,wang2019efficient}).

However, the physiological reason why peptides derived from TMHs 
would be presented more often than peptides 
stemming from soluble protein regions is unknown. 
We hypothesized is that the presentation of 
hydrophobic/TMH residues is evolutionary selected for, 
because TMHs are less prone to undergo escape mutation. 
One reason to expect such a reduced 
variability (and hence evolutionary conservation) in TMHs, 
is that these are restricted in their evolution 
by the functional requirement to span a lipid bilayer. 
Due to these requirements, 
the amino acids genuinely present in TMH 
are limited to the ones with hydrophobic side chains (\cite{jones1994model}).
[FB: Analyzing the occurrence of single nucleotide polymorphisms 
that were predicted to be inside ...]

Therefore, the TMHs of pathogens 
might have a lower chance to develop an escape mutations, 
as many mutations will result in a dysfunctional TMH 
and render the protein inactive.

[GVDB:
 Add one short paragraph, 
 where you reiterate the most important findings:
 In this study, we show that ...(and than basically 
 the abstract in slightly different words and slightly 
 more detail). End with a statement of why it is 
 important (vaccine development based on TMHs?)...
]

%%%%%%%%%%%%%%%%%%%%%%%%%%%%%%%%%%%%%%%%%%%%%%%%%%%%%%%%%%%%%%%%%%%%%%%%%%%%%%%%
\section{Methods}
%%%%%%%%%%%%%%%%%%%%%%%%%%%%%%%%%%%%%%%%%%%%%%%%%%%%%%%%%%%%%%%%%%%%%%%%%%%%%%%%

% \paragraph{Data sets for TMH epitopes}

We used a human, viral and bacterial reference proteome to 
determine the percentages of epitopes overlapping
with TMHs.
We used a human reference proteome
with UniProt ID UP000005640\_9606, which
is a representative set of the full proteome
and used is our previous study (\cite{bianchi2017}).
We used SARS-CoV-2 as our viral reference proteome
with UniProt ID UP000464024 and
Mycobacterium tuberculosis (MTb) 
with UniProt ID UP000001584.
We chose these pathogens based on their impact on the human population.

%%%%%%%%%%%%%%%%%%%%%%%%%%%%%%%%%%%%%%%%%%%%%%%%%%%%%%%%%%%%%%%%%%%%%%%%%%%%%%%%
\subsection{Elution studies}\label{subsec:elution_studies}
%%%%%%%%%%%%%%%%%%%%%%%%%%%%%%%%%%%%%%%%%%%%%%%%%%%%%%%%%%%%%%%%%%%%%%%%%%%%%%%%

To determine if epitopes derived from TMHs are presented at all,
we started from epitopes identified in elution studies
for MHC-I (\cite{schellens2015comprehensive}) 
and MHC-II (\cite{bergseng2015different}).
For each of these peptides, the human representative reference proteome
was searched for its origin.
We kept only the epitopes which were uniquely present
in the proteome.
We predicted the topology of the proteome
and counted how often the epitopes overlapped
with a predicted TMH.
In this analysis, both TMHMM and 
PureseqTM (see section 'Prediction software used') 
were used to predict the topology.
The full analysis can be found
at \url{https://github.com/richelbilderbeek/bbbq_article_issue_157}.

%%%%%%%%%%%%%%%%%%%%%%%%%%%%%%%%%%%%%%%%%%%%%%%%%%%%%%%%%%%%%%%%%%%%%%%%%%%%%%%%
\subsection{Measuring TMH epitopes}
%%%%%%%%%%%%%%%%%%%%%%%%%%%%%%%%%%%%%%%%%%%%%%%%%%%%%%%%%%%%%%%%%%%%%%%%%%%%%%%%

To determine the percentages of MHC-I and MHC-II epitopes overlapping
with TMHs, we used mostly the same analysis as described in \cite{bianchi2017}.
To summarize: from a proteome all possible 9-mers are derived. For each
of these peptides, it was determined if it was part of a 
TMH (at least overlapping with 1 residue), 
and if the peptide bound to an MHC-I haplotype.
For MHC-II, 14-mers were used, as these are the most frequently occurring
epitope length (\cite{bergseng2015different}).

This study differs in some aspects from our previous work (\cite{bianchi2017}), 
described below in more detail.
The main differences are our definition of what a so-called binder is, 
the inclusion of both MHC-I and MHC-II haplotypes, 
the prediction software used, 
and the significance level to determine if TMH-derived peptides will bind.
These deviations are either a refinement of our previous approach or
a pragmatic choice made due to the extension of the original experiment.
Additionally, instead of only using a human proteome, this study
also includes a viral and bacterial proteome.

The definition of a binder differs from \cite{bianchi2017}:
in our current study a peptide was called a binder if, within a certain haplotype, 
any of its 9-mer peptides have an IC50 value in the lowest 2\% of 
the epitopes within a 
\emph{proteome} (see tables \ref{tab:ic50_binders_mhc1} and \ref{tab:ic50_binders_mhc2}
for values), whereas the original study defined
a binder as having an IC50 in the lowest 2\% 
of the peptides within a \emph{protein}.
% See https://github.com/richelbilderbeek/bianchi_et_al_2017/blob/72e6755a31d400158368509fd80a41e984677ab1/predict-binders.R#L17
We believe our revised definition precludes bias of proteins 
that give rise to no or only very few MHC epitopes

The 13 MHC-I haplotypes used in this study are the same as 
we used in our previous study (\cite{bianchi2017}).
The MHC-II haplotypes used additionally were selected 
to occur with a phenotypic frequency of at least 14\% in
the human population (\cite{greenbaum2011functional}),
resulting in 21 haplotypes.
When using an MHC-II haplotype, instead of using 9-mers, 14-mers were
used, as these are the most common MHC-II epitope size,
as found in the elution study (\cite{bergseng2015different}).

%%%%%%%%%%%%%%%%%%%%%%%%%%%%%%%%%%%%%%%%%%%%%%%%%%%%%%%%%%%%%%%%%%%%%%%%%%%%%%%%
\subsubsection{Evolutionary conservation of TMHs}
%%%%%%%%%%%%%%%%%%%%%%%%%%%%%%%%%%%%%%%%%%%%%%%%%%%%%%%%%%%%%%%%%%%%%%%%%%%%%%%%

% \paragraph{Introduction}

To detect the evolutionary conservation of TMHs, 
we collected human mutations and 
tallied their predicted location.
To be more precise, we collected single nucleotide
polymorphisms (SNPs) within the human population
that resulted in the substitution of one amino acid.
We then predicted that protein's topology and tallied if
the SNP occurred within a TMH or not, 
after which we used statistics to determine if there are
more, less or an equal amount of SNPs in TMHs
as a measure of evolutionary conservation.
This workflow is discussed in more detail below.

% \paragraph{Data}

As a data source, multiple
NCBI (\url{https://www.ncbi.nlm.nih.gov/}) databases were used,
which are 'gene', to find the gene names of membrane proteins, 
'dbSNP' (\cite{sherry2001dbsnp}) for SNPs associated with those genes
and 'protein', to obtain the sequence of proteins that SNPs act upon.
The 'dbSNP' contains 650 million 
catalogued non-redundant humane variations (called RefSNPs,
\url{https://www.ncbi.nlm.nih.gov/snp/docs/RefSNP_about/}).

% \paragraph{Data retrieval}

To retrieve the data from these databases the
\verb;rentrez; R package (\cite{rentrez}) was used
that calls the NCBI website's API. To provide for a 
stable user experience for all users, 
this API limits the user to 3 calls per second.
Additionally, the API splits the result of a bigger
query into multiple pages, each of which needs one API call.
We wrote the \verb;sprentrez; package (\cite{sprentrez}) to provide for 
bigger queries of multiple (and delayed) API calls.

% \paragraph{Pipeline}

The first query was a call to the 'gene' database for the 
term 'membrane protein' (in all fields) for the organism \textbf{Homo sapiens}.
This resulted in 1077 gene IDs
[RJCB:
  This was December 2020.
  At 2021-03-01, one will get 1130 gene IDs.
  Also, one of the gene IDs that was valid back then,
  has been obsoleted.
].
The next query was a call to the 'gene' database 
to obtain the gene names from the gene IDs.
Per gene name, the 'dbSNP' NCBI database is queried for 
variations associated with the gene name. 

The number of variations
was limited to the first 250 variations per gene (see figure
\ref{fig:snps_per_gene_name_ncbi} for the number of SNPs
within the dbSNP database for these gene names),
resulting in 60,683 variations at the protein level.
A variation needs not always be a SNP,
as dbSNP also catalogs other DNA alterations, such as, among others, insertions,
deletions and tandem repeats.
We select only the variations that result in a SNP for
a single amino acid substitution, resulting in 38,882 SNPs.

% \paragraph{Selection of SNPs}
%
We selected to analyze SNPs with the lowest SNP ID first.
This means we have a bias for picking SNPs with
an earlier discovery date.
We expect this bias to be biologically irrelevant.
To verify this expectation, we determined the
relative position of each SNP in a protein,
as shown in figure \ref{fig:snp_rel_pos}.

Per SNP, the 'protein' NCBI database was queried for the
protein sequence.
Of each protein sequence, the protein topology was determined 
using PureseqTM.
From the topology and the known location of the SNP, 
we score the location (i.e. TMH or soluble protein region) 
where the change occurred.

%%%%%%%%%%%%%%%%%%%%%%%%%%%%%%%%%%%%%%%%%%%%%%%%%%%%%%%%%%%%%%%%%%%%%%%%%%%%%%%%
\subsection{Prediction software used}
\label{subsec:prediction_software_used}
%%%%%%%%%%%%%%%%%%%%%%%%%%%%%%%%%%%%%%%%%%%%%%%%%%%%%%%%%%%%%%%%%%%%%%%%%%%%%%%%

[RJCB: I suggest to move this section to the Supplementary Materials]

\begin{table}[]
  \begin{tabular}{llll}
    Goal & Tool & Reference \\ 
    \hline
    Predict topology                  & TMHMM                     & \cite{krogh2001predicting} \\
    Predict topology                  & PureseqTM                 & \cite{wang2019efficient} \\
    Predict epitopes MHC-I            & \verb;epitope-prediction; & \cite{bianchi2017} \\
    Predict epitopes MHC-II           & NetMHCIIpan               & \cite{nielsen2008quantitative,karosiene2013netmhciipan} \\
    Call TMHMM from R                 & tmhmm                     & \cite{tmhmm} \\
    Call PureseqTM from R             & pureseqtmr                & \cite{pureseqtmr} \\
    Call NetMHCIIpan from R           & netmhc2pan                & \cite{netmhc2pan} \\
    Combine all                       & bbbq                      & \cite{bbbq}
  \end{tabular}
  \caption{
    Overview of all software used in this research.
  }
  \label{table:software_used}
\end{table}

For this research, the scientific literature was explored 
to identify the most recent free and open source (FOSS) prediction software.
This was done by searching for papers that (1) reference older
prediction software, and (2) present a novel method to make predictions.
As a starting point, a review paper was used.
For all software needed, we found no studies that compared contemporary tools 
in their prediction quality.

% \paragraph{TMH prediction}

There are multiple computational tools developed to predict which
parts of a protein forms a TMH.
In 2001, multiple of such prediction tools have been compared (\cite{moller2001evaluation}),
of which TMHMM (\cite{krogh2001predicting}) turned out to be the best, 
as is used in the previous study (\cite{bianchi2017}).
However, TMHMM has a restrictive software license and is nearly two
decades old.
Therefore, PureseqTM (\cite{wang2019efficient}),
was also used in this study, which has been more recently developed
and has a free software license.

% \paragraph{MHC-I epitope prediction}

For MHC-I, there are multiple computational tools developed 
to predict epitopes. 
According to \cite{lundegaard2011prediction}, at that time,
NetMHCcons (\cite{karosiene2012netmhccons}) gave the best predictions.
We used the same tool as used in our earlier study, \verb;epitope-prediction; (\cite{bianchi2017}),

% \paragraph{MHC-II epitope prediction}

Also for MHC-II, there are multiple computational tools developed 
to predict epitopes,
such as using a trained neural network (\cite{nielsen2003reliable})
or a Gibbs sampling approach (\cite{nielsen2004improved}).
According to \cite{lundegaard2011prediction}, in 2011,
from a set of multiple tools, 
NetMHCIIpan (\cite{nielsen2008quantitative,karosiene2013netmhciipan})
gave rise to the most accurate predictions.
The most recent and promising FOSS tool available now appears
to be MHCnuggets (\cite{shao2020high}), which can do both MHC-I 
and MHC-II prediction, which we used for MHC-II predictions.

%%%%%%%%%%%%%%%%%%%%%%%%%%%%%%%%%%%%%%%%%%%%%%%%%%%%%%%%%%%%%%%%%%%%%%%%%%%%%%%%
\subsection{Prediction software written}
%%%%%%%%%%%%%%%%%%%%%%%%%%%%%%%%%%%%%%%%%%%%%%%%%%%%%%%%%%%%%%%%%%%%%%%%%%%%%%%%

[RJCB: I suggest to move this section to the Supplementary Materials]

The R programming language is used for the complete 
experiment, including the analysis.
The complete experiment is bundled in the 'bbbq' R package,
which is dependent on 'tmhmm', 'pureseqtmr', 
'epitope-prediction' and 'mhcnuggetsr'
as described below.

% \paragraph{tmhmm}

The R package 'tmhmm' was developed to do the similar topology
predictions as our earlier study (that used 'TMHMM'), yet in an automated way.
'TMHMM' has a restrictive software license (\cite{krogh2001predicting}) and allows a user
to download a pre-compiled executable after confirmation he/she
is in academia. The R package respects this restriction
and allows the user to install and use TMHMM from within R,
as done in this study.
'tmhmm' has been submitted to and is accepted by CRAN.

% \paragraph{pureseqtmr}

To be able to call, from R, the TMH prediction 
software 'PureseqTM' (\cite{wang2019efficient}),
which is written in C, the package 'pureseqtmr' has been developed. 
'pureseqtmr' allows to install 'PureseqTM' and use most of its features.
Excluded are the features that are used by the 'PureseqTM' 
developers to verify the correctness of their work.
'pureseqtmr' has been submitted to and is accepted by CRAN.

% \paragraph{mhcnuggetsr}

MHCnuggets is a free and open-source Python package to predict 
epitope affinity for many MHC-I and MHC-II variants (\cite{shao2020high}).
The R package 'mhcnuggetsr' allows one to install and use MHCnuggets
from within R.
Also 'mhcnuggetsr' has been submitted to and is accepted by CRAN.

% \paragraph{bbbq}

To reproduce the full experiment presented in this paper,
the functions needed are bundled in the 'bbbq' R package.
This package is too specific to be submitted to CRAN.

%%%%%%%%%%%%%%%%%%%%%%%%%%%%%%%%%%%%%%%%%%%%%%%%%%%%%%%%%%%%%%%%%%%%%%%%%%%%%%%%
\section{Results}
%%%%%%%%%%%%%%%%%%%%%%%%%%%%%%%%%%%%%%%%%%%%%%%%%%%%%%%%%%%%%%%%%%%%%%%%%%%%%%%%

%%%%%%%%%%%%%%%%%%%%%%%%%%%%%%%%%%%%%%%%%%%%%%%%%%%%%%%%%%%%%%%%%%%%%%%%%%%%%%%%
\subsection{Elution studies}
%%%%%%%%%%%%%%%%%%%%%%%%%%%%%%%%%%%%%%%%%%%%%%%%%%%%%%%%%%%%%%%%%%%%%%%%%%%%%%%%

% tab:elution
% latex table generated in R 4.1.1 by xtable 1.8-4 package
% Fri Oct 29 12:40:25 2021
\begin{table}[ht]
\centering
\begin{tabular}{llll}
  \hline
MHC class & Tool & Dataset & n \\ 
  \hline
I & PureseqTM & schellens & 1.38\% (109/7897) \\ 
  I & PureseqTM & iedb & 6.81\% (43/631) \\ 
  I & TMHMM & schellens & 1.43\% (113/7897) \\ 
  I & TMHMM & iedb & 7.13\% (45/631) \\ 
  II & PureseqTM & bergseng & 3.92\% (498/12712) \\ 
  II & PureseqTM & iedb & 0.29\% (4/1364) \\ 
  II & TMHMM & bergseng & 3.96\% (504/12712) \\ 
  II & TMHMM & iedb & 1.39\% (19/1364) \\ 
   \hline
\end{tabular}
\caption{Percentage of epitopes derived from a TMH found in the two elution studies, for the two different kind of topology prediction tools. The values between braces show the the number of epitopes that were predicted to overlapping with a TMH per all epitopes that could be uniquely mapped to the representative human reference proteome.} 
\label{tab:elution}
\end{table}


To confirm that peptides stemming from TMHs were presented in MHC-I and MHC-II,
we reanalyzed published peptide elution studies.
Table \ref{tab:elution} shows the percentage of epitopes derived 
from a TMH
found in MHC-I and MHC-II elution 
studies (\cite{schellens2015comprehensive} and \cite{bergseng2015different} respectively),
for the two topology prediction tools TMHMM and 
PureseqTM (\cite{krogh2001predicting} and \cite{wang2019efficient} respectively). 
Regardless of the prediction, 
at least 100 epitopes were predicted to be derived from a TMH. 
From these findings, it is robustly predicted that
epitopes derived from TMHs are presented by B cells [For Frans: please check
that the article by Bergseng, 2015, 'Different binding motifs of the celiac 
disease-associated HLA molecules DQ2. 5, DQ2. 2, and DQ7. 5 revealed by 
relative quantitative proteomics of endogenous peptide repertoires'
is appropriate. As far as I can see, it is, but I may overlook something. Richel].


%%%%%%%%%%%%%%%%%%%%%%%%%%%%%%%%%%%%%%%%%%%%%%%%%%%%%%%%%%%%%%%%%%%%%%%%%%%%%%%%
\subsection{MHC-I}
%%%%%%%%%%%%%%%%%%%%%%%%%%%%%%%%%%%%%%%%%%%%%%%%%%%%%%%%%%%%%%%%%%%%%%%%%%%%%%%%

We recently showed that peptides derived from TMHs of human proteins 
were predicted to be over-presented in MHC-I. 
Here, we wondered if this finding would be general 
and if peptides stemming from TMHs of microbial pathogens 
would also be over-presented.
Figure \ref{fig:1} shows the percentages of MHC-I epitopes 
predicted to be overlapping 
with TMHs for our human, viral and bacterial proteome.
See the supplementary materials (table \ref{tab:tmh_binders_mhc1}) 
for the exact TMH and epitope counts.
We found that also for pathogen epitopes derived from TMHs are
over-presented. 
The extent of the over-presentation is similar between the haplotypes:
if a haplotype if prone to present TMH-derived epitopes from
human proteins, it is also prone to present such epitopes of
bacterial and viral proteins.

\begin{figure}[!htbp]
  \includegraphics[height=0.9\textheight]{bbbq_1_smart_results/fig_f_tmh_mhc1_2_grid.png}
  \caption{
    Percentage of MHC-I epitopes overlapping with TMHs
    for a human, viral and bacterial proteome.
    The horizontal lines indicate the percentage as expected by chance.
    See table \ref{tab:tmh_binders_mhc1} for the exact TMH and epitope counts.
  }
  \label{fig:1}
\end{figure}

%%%%%%%%%%%%%%%%%%%%%%%%%%%%%%%%%%%%%%%%%%%%%%%%%%%%%%%%%%%%%%%%%%%%%%%%%%%%%%%%
\subsection{MHC-II}
%%%%%%%%%%%%%%%%%%%%%%%%%%%%%%%%%%%%%%%%%%%%%%%%%%%%%%%%%%%%%%%%%%%%%%%%%%%%%%%%

Where TMH-derived epitopes are over-presented by MHC-I,
we wondered if this finding would also generalize to MHC-II.
Figure \ref{fig:2} shows the percentages of MHC-II epitopes overlapping 
with TMHs for our human, viral and bacterial proteome.
See the supplementary materials (table \ref{tab:tmh_binders_mhc2}) 
for the TMH and epitope counts.
We show that MHC-II has the same pattern of over-presentation
as MHC-I, yet this over-presentation is more extreme.
Also for MHC-II, we observe that the extent the which TMH-derived
epitopes are presented is similar between the haplotypes, for
a human, bacterial and viral proteome.

\begin{figure}[!htbp]
  \includegraphics[height=0.9\textheight]{bbbq_1_smart_results/fig_f_tmh_mhc2_2_grid.png}
  \caption{
    Percentage of MHC-II epitopes overlapping with TMHs
    for a human, viral and bacterial proteome.
    The horizontal lines indicate the percentage as expected by chance.
    See table \ref{tab:tmh_binders_mhc2} for the exact TMH and epitope counts.
    Note that for smaller proteomes a percentage of zero is likelier.
  }
  \label{fig:2}
\end{figure}

%%%%%%%%%%%%%%%%%%%%%%%%%%%%%%%%%%%%%%%%%%%%%%%%%%%%%%%%%%%%%%%%%%%%%%%%%%%%%%%%
\subsection{Evolutionary conservation}
%%%%%%%%%%%%%%%%%%%%%%%%%%%%%%%%%%%%%%%%%%%%%%%%%%%%%%%%%%%%%%%%%%%%%%%%%%%%%%%%

\begin{figure}[!htbp]
  \includegraphics[width=\textwidth]{ncbi_peregrine_results/fig_conservation.png}
  \caption{
    Evolutionary conservation of SNPs in TMHs.
  }
  \label{fig:conservation}
\end{figure}

\begin{figure}[!htbp]
  \includegraphics[width=\textwidth]{ncbi_peregrine_results/fig_conservation_per_spanner.png}
  \caption{
    Evolutionary conservation of SNPs in TMHs,
    for either single- or multi-spanners
  }
  \label{fig:conservation_per_spanner}
\end{figure}

\begin{figure}[!htbp]
  \includegraphics[width=\textwidth]{ncbi_peregrine_results/fig_f_snps_found_and_expected.png}
  \caption{
    Percentage of SNPs found in TMHs.
    Each point resembles one protein, with a predicted percentage of
    TMH (x-axis) and an observed occurrence of SNPs being located
    within a TMH (y-axis).
    The dashed diagonal line shows the expected linear trend line
    if TMHs and soluble protein regions are equally conserved.
    The red line is a linear trend line that includes the membrane-associated
    proteins at the origin. 
    The blue line is a linear trend line that includes only the
    transmembrane proteins.
  }
  \label{fig:f_snps_found_and_expected}
\end{figure}

In the final set of experiments, 
we addressed the question whether TMHs 
would be more evolutionary conserved than soluble proteins regions 
by comparing the occurrences of SNPs in the genome coding 
for TMHs and soluble protein regions within membrane proteins.
We obtained 1,129 [For Frans: I would enjoy to write 'approx 1k' here, would that
be acceptable? I do refer to a table with the exact values
later on anyways. Richel] gene names associated with the phrase 'membrane protein',
which are genes for both membrane-associated proteins (MAPs, no TMH) and 
transmembrane proteins (TMPs, at least one TMH).
These genes are linked to 4,780 protein isoforms, 
of which 2,553 are predicted to be TMPs (hence 
2,237 proteins are predicted to be MAPs).
We obtained 37,630 SNPs that resulted in an
amino acids substitution, of which 21,024 were located 
in predicted TMPs.
See table \ref{tab:ncbi_counts} for an overview of all amounts.

Per protein, we calculated two percentages: 
(1) the percentage of the protein coding for TMHs, 
and (2) the percentage of SNPs occurring in TMHs.
Each percentage pair was plotted in figure \ref{fig:f_snps_found_and_expected}.
We performed a linear regression analyses on the data twice,
one with and one without MAPs,
and added a 95\% confidence interval for both of these regressions.
Would the confidence interval of the regression line 
overlap with the line of equality,
this would be expected if SNPs occur just as likely within TMHs 
and soluble protein domains.
This analysis gives the first hint that SNPs 
occur more often in soluble domains than in TMHs.

To statistically test whether SNPs occur 
just as likely in soluble domains as
in TMHs, we first calculated the number of SNPs that are expected to be
found in TMHs of TMPs by chance, $E(n_{SNP\_TMH})$,
equals the number of SNPs in TMHs, $n_{SNP} = 21,738$, 
multiplied by the fraction of the TMPs that are TMH, $f_{TMH} = 18.66\%$,
$E(n_{SNP\_TMH}) = n_{SNP} \cdot f_{TMH} \approx 4056$.

The effect size is that, 
when randomly selecting SNPs, 
per 100 SNPs found in soluble protein domains, 
one finds 91 SNPs in TMHs,
as depicted in figure \ref{fig:conservation}.

The actually number of SNPs found in TMHs
was $n_{SNP\_TMH} = 3,903$, which is lower than the expectation
of $E(n_{SNP\_TMH}) \approx 4056$.
To test whether this difference is significant,
we used a binomial Poisson distribution, where the number of trails,
$n$ equals the number of SNPs, which is 21,738. The probability of success
for TMP $i$, $p\_i$, is the percentage of each TMP that is TMH.
We calculated the probability of having 3,903 or less successes.
We used the type I error cut-off value of $\alpha = 2.5\%$.
The chance to find, within TMHs, this amount or less SNPs 
equals [RJCB: recalculate]
$1.6201 \cdot 10^{-9}$.

Using a alpha value of $0.05$, as is common practice for an uninformed
expectation, we conclude that there are significantly less SNPs
in TMHs as expected by chance.

[RJCB: How many SNPs in TMHs?]
See figure \ref{fig:f_tmh_ncbi} to see which percentages
are TMH of the transmembrane proteins.

\begin{figure}[!htbp]
  \includegraphics[width=\textwidth]{ncbi_peregrine_results/fig_f_snps_found_and_expected_per_spanner.png}
  \caption{
    Percentage of SNPs found in TMHs per spanner.
    The dashed diagonal lines show the expected linear trend line
    if TMHs and soluble protein regions are equally conserved.
  }
  \label{fig:f_snps_found_and_expected_per_spanner}
\end{figure}
)

We decided to split this analysis based on the number of TMHs
a protein has 
[GVDB: Why? Provide brief rationale again. 
Thus TMH-TMH interactions or something]. 
We hypothesize that single-spanners (i.e. proteins
with one TMH predicted) are less conserved, when compared to multi-spanners,
as single spanners just need to span a membrane, while multi-spanners
might have interactions between their TMHs, 
for example to accommodate active sites, and 
thus might have additional structural constraints.
As can be seen in figure \ref{fig:f_snps_found_and_expected_per_spanner}, 
both single- and multi-spanners are evolutionarily conserved.

We conclude that,
when randomly selecting SNPs, 
for 100 SNPs found in soluble regions of multi-spanners,
one finds 89.6 SNPs in TMHs, as shown by 
figure \ref{fig:conservation_per_spanner}.

%%%%%%%%%%%%%%%%%%%%%%%%%%%%%%%%%%%%%%%%%%%%%%%%%%%%%%%%%%%%%%%%%%%%%%%%%%%%%%%%
\section{Conclusion}
%%%%%%%%%%%%%%%%%%%%%%%%%%%%%%%%%%%%%%%%%%%%%%%%%%%%%%%%%%%%%%%%%%%%%%%%%%%%%%%%

% \paragraph{General pattern in epitope presentation}

From this study, two important conclusions can be drawn. 
First, the MHC over-presentation of TMHs is likely a general feature 
and predicted to occur for most haplotypes of both MHC-I and -II 
and for humans as well as bacterial and viral pathogens. 
Second, TMHs are more evolutionary conserved that soluble protein motifs, 
at least in the human proteome. 

% \paragraph{Over-presentation in vivo in MHC-I is robust}

We confirm that 1.4\% of epitopes that are naturally presented on MHC-I 
can be mapped to stem from a TMH. Additionally, this 
prediction is robust, as it holds for two TMH prediction tools, 
as shown by table \ref{tab:elution}).

% \paragraph{Over-presentation in vivo in MHC-II is robust}

We predict that also MHC-II presents epitopes derived from TMHs
in vivo, as we find that 3.9-4.0\% of sequenced epitopes
can be uniquely traced onto a human reference 
proteome (also shown in table \ref{tab:elution}).

% \paragraph{Over-presentation in silico}

Not only do we predict that MHC-I and MHC-II TMH-derived epitopes 
are presented in vivo, we conclude that these are additionally
presented more often than expected chance 
alone (figures \ref{fig:1} and \ref{fig:2}).
This in silico finding is a general pattern 
that persists in both MHC-I and MHC-II, 
for humans and two pathogens.

% \paragraph{Evolutionary conservation}

We conclude that mutations are more likely to occur
in soluble regions of proteins, than in TMHs.
When selecting SNPs at random, per 1000 SNPs found in a soluble domain, 
one finds on average 918 SNPs in TMHs.
This effect is present in TMPs with two or more TMHs, 
but not in single-spanners.

%%%%%%%%%%%%%%%%%%%%%%%%%%%%%%%%%%%%%%%%%%%%%%%%%%%%%%%%%%%%%%%%%%%%%%%%%%%%%%%%
\section{Discussion}
%%%%%%%%%%%%%%%%%%%%%%%%%%%%%%%%%%%%%%%%%%%%%%%%%%%%%%%%%%%%%%%%%%%%%%%%%%%%%%%%

[GVDB: 
  Discussion section needs some work, 
  as here interpretation of the results needs to be given. 
  What is the consequence of the findings? Why is it important. 

  Now it is just a list of potential reasons why the results might be flawed. 
  While this is in principle OK, 
  it is also nitpicking as some things affect only a limited number of 
  proteins (like selenoproteins) 
  or you’d expect changes in the other direction (wrong prediction of TMHs for bacteria). 

  So offer interpretations. 
  The second part of the introduction would fit here well. I can also help with this.
]

% \paragraph{General}

Our understanding of the immune system and epitope prediction is 
important to fight disease. By this paper, we suggest that epitopes
derived from TMHs are needlessly overlooked and may inspire to investigate
new potential targets for combating pathogens.

\subsection{Elution studies}

% \paragraph{TMH presentation in vivo}

From elution studies and in silico predictions, 
we suggest that TMH-derived epitopes are presented in vivo.
The biochemical intracellular pathway to achieve this,
however, is still unknown. 

% \paragraph{False positives}

An additional unknown is, although our findings are robust,
whether this is a false positive,
due to additive methodological artifacts.
An obvious artifact is that the elution studies used 
will contain some unavoidable sequencing errors.
Additionally, we assume that the epitopes naturally presented
are derived from a cleanly translated proteome.
Two known biological phenomena that violate this assumption
are defective ribosomal products (DRiPs) (\cite{yewdell1996defective}) 
and peptide fusion (\cite{delong2016pathogenic}),
two processes that both result in peptide fragments
that did not originate from genome directly.
Also, the TMH prediction software is imperfect.
This imperfection, however, appears to be of limited importance,
as two different topology prediction software have similar results.
Although those a couple of percent of epitopes that 
can be mapped directly to the proteome,
there is methodological bias caused by mass spectronomy,
as hydrophobic fragments are detected less 
by mass spectronomy. This means that the number of TMH-derived epitopes
is underestimated.
in vivo work in the lab will be needed to resolve this finding.

\subsection{Presentation of epitopes derived from TMHs}

% \paragraph{MHC-II is expected to present TMHs}

This study predict that TMH-derived epitopes are over-presented
in MHC-II as well. 
The intracellular pathway of MHC-II differs from that of MHC-I,
as it acquires epitopes from the exterior of the cell by phagocytosis.
How TMHs derived from exterior TMPs are lysed and loaded onto MHC-II 
is unknown. in vivo work is needed to elucidate this novel pathway.

% \paragraph{Immunogenicity}

In this study, we find that TMH-derived epitopes are over-presented.
The presentation of an epitope, however, 
is only a first step of an adaptive immune response.
We assume that the presentation of epitopes is
correlated with evoking a full immune response.
TMH-derived epitopes, however, are more hydrophobic than the average
epitope. If this evokes a stronger or weaker immune response is unknown,
and in vivo work is needed to resolve this.

\subsection{Evolutionary conservation of TMHs}

% \paragraph{Selection undetectable in whole proteome}

[RJCB: this paragraph I ruthlessly moved
from introduction to here, as was suggested by GVDB]
 
In general, on would hope that evolutionary selection results in
an immune system that as most attentive for loci that are
essential for a virus, as these will be most conserved.
In SARS-CoV-2, for example, there is preliminary evidence that the strongest
selection pressure is upon residues that changes its 
virulence (\cite{velazquez2020positive}).
These loci, however, only account for a small part of a pathogen's proteome.
Additionally, these essential parts differ widely between pathogens.
Because of this scarcity and variance in targets, 
one can imagine that the human immune system 
is not tailored to detect these sites, 
 as hinted by upon by the aforementioned influenza study.
 
% \paragraph{Selection may be detectable in TMHs}

[RJCB: this paragraph I ruthlessly moved
from introduction to here, as was suggested by GVDB]
 
TMHs, on the other hand, also have their function constraints, 
yet can occur multiple time a pathogen's proteome.
One can safely assume a pathogen's proteome contains multiple TMHs.
Therefore, it may be beneficial for the host
if its immune system would be more attentive towards TMHs.
And maybe this has already happened: MHC-I already detects hydrophobic
peptides. This feature, however, may also be caused by selection
to detect hydrophobic regions in the soluble proteins of pathogens.
It is unknown, when focusing on TMHs only, if a signal of selection
can be detected.






\subsection{Evolutionary conservation}

[GVDB: 
  This might be relevant, but you need to explain why it is relevant. 
  Thus compare your findings with previous efforts 
  to look at conservation of TMHs
]
% \paragraph{synonymous mutations}
%
In our evolutionary experiment, 
we removed variations that were synonymous mutations (i.e.
resulted in the same amino acid, from a different genetic code) 
from our analysis.
There is evidence, however, that these synonymous mutations
do have an effect and may even be evolutionary selected 
for (\cite{hunt2009silent}).
As the possible effect of synonymous mutations is ignored by our
topology prediction software, we do so as well.



% \paragraph{Low number of SNPs}

The NCBI dbSNP database contains millions of SNPs.
As the NCBI API constrains its users to three calls per second,
we had to limited the extent of our analysis.

[RJCB: this paragraph I ruthlessly moved
from introduction to here, as was suggested by GVDB]
We concluded that the
epitopes that MHC-I presents are [as/not as] likely 
to be derived from TMH within either a human host and its bacterial pathogen.
Because a bacterium does not infect a cell, thus its peptides
will not be presented by MHC-I, this result is [unexpected/expected]

[RJCB: this paragraph I ruthlessly moved
from introduction to here, as was suggested by GVDB]
We aimed our evolutionary experiment at TMHs, because these can
be predicted well from a protein structure,
are common structures and are present in all pathogens. 
We could have done the same experiment on beta-turn,
as also these can be predicted well (\cite{petersen2010netturnp}),
are common structures and are present in all pathogens.

[RJCB: this paragraph I ruthlessly moved
from introduction to here, as was suggested by GVDB]
The human immune system and human pathogen are in an evolutionary
arms race: our immune systems is selected for the detection
of pathogens, whereas pathogens are selected to avoid detection.
From a pathogen's point of view, however, this struggle 
is of only minor importance:
in seasonal influenza, for example, the selection pressure
exerted by the immune system was only limited (\cite{han2019individual}).

% \paragraph{Interesting haplotype for SARS-CoV-2}

It is known that an individual's haplotype
influences the dynamics of a pathogen.
For example, in the fast-mutating HIV virus, 
HLA haplotypes that are likelier to induce a specific T-cell response
for the Gag protein slow down the disease's 
progression \cite{eccleston2017host}.
The bioinformatics pipeline developed in this
study may be interesting to make in silico
predictions of a SARS-CoV-2 infection.
The most prominent example is the (MHC-I)
haplotype HLA-A*24:02,
which is predicted to present four times as much epitopes derived 
from SARS-CoV-2 TMHs than expected by chance.
This may help the immune system to detect a SARS-CoV-2 infection earlier,
and change the infection's dynamics.

%%%%%%%%%%%%%%%%%%%%%%%%%%%%%%%%%%%%%%%%%%%%%%%%%%%%%%%%%%%%%%%%%%%%%%%%%%%%%%%%
\section{Acknowledgments}
%%%%%%%%%%%%%%%%%%%%%%%%%%%%%%%%%%%%%%%%%%%%%%%%%%%%%%%%%%%%%%%%%%%%%%%%%%%%%%%%

We thank the Center for Information Technology of the University 
of Groningen for its support and for providing access to the Peregrine 
high performance computing cluster. 
Additionally, we would like to thank Sci-Hub (\cite{himmelstein2018sci})
for allowing us to read paywalled articles while working from home.
[GVDB: 
  Should we include Maxim? 
  He did the alignment of Mycobacterium tuberculosis. 
  Let’s also ask Frans.
]

%%%%%%%%%%%%%%%%%%%%%%%%%%%%%%%%%%%%%%%%%%%%%%%%%%%%%%%%%%%%%%%%%%%%%%%%%%%%%%%%
\section{Data Accessibility}
%%%%%%%%%%%%%%%%%%%%%%%%%%%%%%%%%%%%%%%%%%%%%%%%%%%%%%%%%%%%%%%%%%%%%%%%%%%%%%%%

All code is archived at \url{http://github.com/richelbilderbeek/someplace},
with DOI \url{https://doi.org/12.3456/zenodo.1234567}.

%%%%%%%%%%%%%%%%%%%%%%%%%%%%%%%%%%%%%%%%%%%%%%%%%%%%%%%%%%%%%%%%%%%%%%%%%%%%%%%%
\section{Authors' contributions}
%%%%%%%%%%%%%%%%%%%%%%%%%%%%%%%%%%%%%%%%%%%%%%%%%%%%%%%%%%%%%%%%%%%%%%%%%%%%%%%%

RJCB and FB conceived the idea for this research. 
RJCB wrote the code.
RJCB and FB wrote the article.

%%%%%%%%%%%%%%%%%%%%%%%%%%%%%%%%%%%%%%%%%%%%%%%%%%%%%%%%%%%%%%%%%%%%%%%%%%%%%%%%
% Bibliography
%%%%%%%%%%%%%%%%%%%%%%%%%%%%%%%%%%%%%%%%%%%%%%%%%%%%%%%%%%%%%%%%%%%%%%%%%%%%%%%%
% MEE style
\bibliographystyle{mee}
\bibliography{bbbq_article}
%%%%%%%%%%%%%%%%%%%%%%%%%%%%%%%%%%%%%%%%%%%%%%%%%%%%%%%%%%%%%%%%%%%%%%%%%%%%%%%%


%%%%%%%%%%%%%%%%%%%%%%%%%%%%%%%%%%%%%%%%%%%%%%%%%%%%%%%%%%%%%%%%%%%%%%%%%%%%%%%%
\appendix
\section{Supplementary materials}
%%%%%%%%%%%%%%%%%%%%%%%%%%%%%%%%%%%%%%%%%%%%%%%%%%%%%%%%%%%%%%%%%%%%%%%%%%%%%%%%

%%%%%%%%%%%%%%%%%%%%%%%%%%%%%%%%%%%%%%%%%%%%%%%%%%%%%%%%%%%%%%%%%%%%%%%%%%%%%%%%
\subsection{Differences with Bianchi et al., 2017}
%%%%%%%%%%%%%%%%%%%%%%%%%%%%%%%%%%%%%%%%%%%%%%%%%%%%%%%%%%%%%%%%%%%%%%%%%%%%%%%%

A part of this study does the same analysis as Bianchi et al., 2017.
Here we describe the deviations, which are about the use of different
software and the use of a different definition of what a binder is.

% Percentage of spots and spots that overlap with a TMH
\input{bbbq_1_smart_results/table_f_tmh_2.latex}

% \paragraph{IC50 prediction software}

The earlier study uses \verb;epitope-prediction; a hand-crafted method, 
that has been trained on MHC-I haplotypes only,
which is used here again. For MHC-II IC50 predictions, the
'mhcnuggetsr' R package is used.

% \paragraph{Definition of what a binder is}

The earlier study defines a peptide a binder (for a haplotype), 
if \emp{within the peptide} in which it is found, 
is within the peptides with the 2\% lowest IC50 values.
This can be seen at \url{https://github.com/richelbilderbeek/bianchi_et_al_2017/blob/master/predict-binders.R},
where the binders are written to file.

In this study, a peptide is defined a binder (for a haplotype), 
if within a \emp{proteome} in which it is found, 
is within the peptides with the 2\% lowest IC50 values,
where our previous study used the lowest 2\% IC50 values
within each \emp{protein}.
Subsection \ref{subsec:ic50s_per_haplotype} shows the IC50 values
for a binder per haplotype.

% \paragraph{Selenoproteins}

Our previous study used the TMHMM web server
to predict TMHs.
The desktop version of TMHMM, however, gives an
error message on the 25 selenoproteins found in the human
reference proteome.
For the sake of reproducible research, we use the desktop version (as
we can call it from scripts) and, due to this, we removed the
selenoproteins from this analysis.

%%%%%%%%%%%%%%%%%%%%%%%%%%%%%%%%%%%%%%%%%%%%%%%%%%%%%%%%%%%%%%%%%%%%%%%%%%%%%%%%
\subsection{Minor methods}
%%%%%%%%%%%%%%%%%%%%%%%%%%%%%%%%%%%%%%%%%%%%%%%%%%%%%%%%%%%%%%%%%%%%%%%%%%%%%%%%

PureseqTM does not predict the topology
of proteins that have less than three amino acids. 
The TRDD1 ('T cell receptor delta diversity 1') protein,
however, is two peptides long. 
The R package \verb;pureseqtmr;, however, 
predicts that mono- and di-peptides are cytosolic.

%%%%%%%%%%%%%%%%%%%%%%%%%%%%%%%%%%%%%%%%%%%%%%%%%%%%%%%%%%%%%%%%%%%%%%%%%%%%%%%%
\subsection{Minor discussion}
%%%%%%%%%%%%%%%%%%%%%%%%%%%%%%%%%%%%%%%%%%%%%%%%%%%%%%%%%%%%%%%%%%%%%%%%%%%%%%%%

% \paragraph{Bacteria have a different cell membrane}

In this experiment we predicted epitopes that overlap with 
TMHs from a human, bacterial and viral proteome,
would these proteins be expressed in a human host.
Bacteria, however have different cell membranes and cell walls, 
hence different structural requirements for a TMH.
Both topology prediction tools were trained to recognize
human TMHs, thus we cannot be sure that
the transmembrane regions predicted in bacterial proteins
are actually part of a TMH.
For the purpose of this study, we assume the 
error in topology predictions to be unbiased way towards topology.
In other words: that a bacterial TMH is incorrectly
predicted to be absent just as often as it is incorrectly
predicted to be present elsewhere.

% \paragraph{False positives in SNPs}

Regarding the evolutionary conservation of TMHs using SNPs,
again, it is estimated that approximately ten percent
of SNPs is a false positive that result from the methods to determine
a SNP. One example is that sequence variations are incorrectly
detected due to highly similar duplicated sequences \cite{musumeci2010single}.
We assume that these duplications occur as often in TMHs as in
regions around these, hence we expect this not to affect our results.

%%%%%%%%%%%%%%%%%%%%%%%%%%%%%%%%%%%%%%%%%%%%%%%%%%%%%%%%%%%%%%%%%%%%%%%%%%%%%%%%
\subsection{IC50s per haplotype}
\label{subsec:ic50s_per_haplotype}
%%%%%%%%%%%%%%%%%%%%%%%%%%%%%%%%%%%%%%%%%%%%%%%%%%%%%%%%%%%%%%%%%%%%%%%%%%%%%%%%

Per target proteome (i.e. human, SARS-CoV-2, Mycobacterium tuberculosis),
we collected all 9-mers (for MHC-I) and 14-mers (for MHC-II),
after removing the selenoproteins and proteins that are shorter
than the epitope length.
From these epitopes, per MHC haplotype,
we predicted the IC50 (in nM) using \verb;epitope-prediction; (for MHC-I)
and MHCnuggets (for MHC-II). 
Here, we show the IC50 value per haplotype that
is used to determine if a peptide binds to the haplotype's MHC
for MHC-I (see table \ref{tab:ic50_binders_mhc1}) and 
MHC-II (see table \ref{tab:ic50_binders_mhc2}).

% tab:ic50_binders_mhc1
\input{bbbq_1_smart_results/table_ic50_binders_mhc1_2.latex}

% tab:ic50_binders_mhc2
\input{bbbq_1_smart_results/table_ic50_binders_mhc2_2.latex}

%%%%%%%%%%%%%%%%%%%%%%%%%%%%%%%%%%%%%%%%%%%%%%%%%%%%%%%%%%%%%%%%%%%%%%%%%%%%%%%%
\subsection{MHC-I}
%%%%%%%%%%%%%%%%%%%%%%%%%%%%%%%%%%%%%%%%%%%%%%%%%%%%%%%%%%%%%%%%%%%%%%%%%%%%%%%%

\begin{figure}[!htbp]
  \includegraphics[width=\textwidth]{bbbq_1_smart_results/fig_f_tmh_mhc1_2_normalized.png}
  \caption{
    Normalized proportion of MHC-I epitopes overlapping with TMHs
    for human, viral and bacterial proteomes.
    Legend: covid = SARS-CoV-2,
    human = homo sapiens, myco = Mycobacterium tuberculosis
  }
  \label{fig:f_tmh_mhc1_normalized}
\end{figure}

% Label: tab:tmh_binders_mhc1
\input{bbbq_1_smart_results/table_tmh_binders_mhc1_2.latex}

%%%%%%%%%%%%%%%%%%%%%%%%%%%%%%%%%%%%%%%%%%%%%%%%%%%%%%%%%%%%%%%%%%%%%%%%%%%%%%%%
\subsection{MHC-II}
%%%%%%%%%%%%%%%%%%%%%%%%%%%%%%%%%%%%%%%%%%%%%%%%%%%%%%%%%%%%%%%%%%%%%%%%%%%%%%%%

\begin{figure}[!htbp]
  \includegraphics[width=\textwidth]{bbbq_1_smart_results/fig_f_tmh_mhc2_2_normalized.png}
  \caption{
    Normalized proportion of MHC-II epitopes overlapping with TMHs
    for human, viral and bacterial proteomes.
    Legend: covid = SARS-CoV-2,
    human = homo sapiens, myco = Mycobacterium tuberculosis
  }
  \label{fig:f_tmh_mhc2_normalized}
\end{figure}

% Label: tab:tmh_binders_mhc2
\input{bbbq_1_smart_results/table_tmh_binders_mhc2_2.latex}

%%%%%%%%%%%%%%%%%%%%%%%%%%%%%%%%%%%%%%%%%%%%%%%%%%%%%%%%%%%%%%%%%%%%%%%%%%%%%%%%
\subsection{Evolutionary conservation}
%%%%%%%%%%%%%%%%%%%%%%%%%%%%%%%%%%%%%%%%%%%%%%%%%%%%%%%%%%%%%%%%%%%%%%%%%%%%%%%%

See tables \ref{tab:ncbi_counts_1} and tables \ref{tab:ncbi_counts_2}
for an overview of all amounts.
In table \ref{tab:ncbi_counts_1} one expects [RJCB: name these]
numbers to add up. 
The number of unique SNPs and unique gene names
does not add up for MAP and TMP,
as one SNP may work on multiple isoforms, some of which can be MAP
where others can be TMP.
The number of unique SNPs and unique gene names
does not add up for TMPs in TMH and TMP in soluble regions
as one SNP may work on multiple isoforms, some of which can be MAP
where others can be TMP, and some SNPs fall in a TMH, where others
are found in soluble regions
In table \ref{tab:ncbi_counts_2} one expects [RJCB: name these]
numbers to add up. 


% Label: tab:ncbi_counts_1
\begin{table}

\caption{\label{tab:ncbi_counts}Amounts. raw = all variations, including DNA variations. all\_proteins = all proteins. map = membrane associated protein. tmp = transmembrane protein. in\_tmh = in transmembrane helix of TMP. in\_sol = in soluble region of TMP. }
\centering
\begin{tabular}[t]{l|r|r|r|r|r|r}
\hline
what & raw & all\_proteins & map & tmp & in\_tmh & in\_sol\\
\hline
Number of variations & 60931 & 37831 & 16623 & 21208 & 3803 & 17405\\
\hline
Number of unique variations & 60544 & 37630 & 16606 & 21024 & 3789 & 17235\\
\hline
Number of unique SNPs & NA & 9621 & 4219 & 6026 & 1140 & 4936\\
\hline
Number of unique gene names & 953 & 911 & 457 & 605 & 325 & 590\\
\hline
Number of unique protein names & 5163 & 4780 & 2227 & 2553 & 1280 & 2467\\
\hline
Percentage TMH & NA & 10 & 0 & 19 & 26 & 18\\
\hline
\end{tabular}
\end{table}

% Label: tab:ncbi_counts_2
\begin{table}

\caption{\label{tab:ncbi_counts_2}Amounts. single\_in\_tmh = in transmembrane helix of single-spanner. single\_in\_sol = in soluble region of single-spanner. multi\_in\_tmh = in transmembrane helix of multi-spanner. multi\_in\_sol = in soluble region of multi-spanner. }
\centering
\begin{tabular}[t]{l|r|r|r|r}
\hline
what & single\_in\_tmh & single\_in\_sol & multi\_in\_tmh & multi\_in\_sol\\
\hline
Number of variations & 452 & 7734 & 3351 & 9671\\
\hline
Number of unique variations & 451 & 7733 & 3338 & 9502\\
\hline
Number of unique SNPs & 160 & 2393 & 994 & 2762\\
\hline
Number of unique gene names & 96 & 282 & 243 & 344\\
\hline
Number of unique protein names & 304 & 1032 & 976 & 1435\\
\hline
Percentage TMH & 11 & 5 & 35 & 26\\
\hline
\end{tabular}
\end{table}

Figure \ref{fig:snps_per_gene_name_ncbi} shows the distribution of the
number of SNPs per gene name, at 2020-12-14.

\begin{figure}[!htbp]
  \includegraphics[width=\textwidth]{ncbi_peregrine_results/fig_snps_per_gene_name_ncbi.png}
  \caption{
    Distribution of the number of SNPs per gene name in the NCBI database.
  }
  \label{fig:snps_per_gene_name_ncbi}
\end{figure}

\begin{figure}[!htbp]
  \includegraphics[width=\textwidth]{ncbi_peregrine_results/fig_snps_per_gene_name_processed.png}
  \caption{
    Distribution of the number of protein variations and SNPs per gene name processed.
  }
  \label{fig:snps_per_gene_name_processed}
\end{figure}

Figure \ref{fig:f_tmh_ncbi} shows the distribution 
of the percentages of TMH of transmembrane proteins.

\begin{figure}[!htbp]
  \includegraphics[width=\textwidth]{ncbi_peregrine_results/fig_f_tmh_ncbi.png}
  \caption{
    Distribution of the percentages of TMH of transmembrane proteins.
    Dashed vertical line: mean \% TMH in all proteins.
    Dotted vertical line: mean \% TMH in TMPs.
  }
  \label{fig:f_tmh_ncbi}
\end{figure}

\begin{figure}[!htbp]
  \includegraphics[width=\textwidth]{ncbi_peregrine_results/fig_n_proteins_per_gene_name.png}
  \caption{
    Histogram of the number of proteins found per gene name.
    Most often, a gene name is associated with one proteins. 
  }
  \label{fig:n_proteins_per_gene_name}
\end{figure}


\begin{figure}[!htbp]
  \includegraphics[width=\textwidth]{ncbi_peregrine_results/fig_n_snps_per_tmp.png}
  \caption{
    Histogram of the number of SNPs per trans-membrane protein.
    Dashed vertical line: average number of SNPs per TMP
  }
  \label{fig:n_tmhs_per_protein}
\end{figure}

\begin{figure}[!htbp]
  \includegraphics[width=\textwidth]{ncbi_peregrine_results/fig_n_tmhs_per_protein.png}
  \caption{
    Histogram of the number of TMHs predicted per protein,
    for the trans-membrane proteins used.
  }
  \label{fig:n_tmhs_per_protein}
\end{figure}

\begin{figure}[!htbp]
  \includegraphics[width=\textwidth]{ncbi_peregrine_results/fig_snp_rel_pos.png}
  \caption{
    Distribution of the relative position of the SNPs used,
    where a relative position of zero denotes the first amino
    acid at the N-terminus, where a relative position of one
    indicates the last residue at the C-terminus.
  }
  \label{fig:snp_rel_pos}
\end{figure}

% \paragraph{Relative position of SNPs}

To verify if SNPs were sampled uniformly
over proteins, we show the distribution 
of the relative position in figure \ref{fig:snp_rel_pos}.
We find no clear evidence of a bias.

% \paragraph{Statistics}

Table \ref{tab:snp_stats} shows the statistics for all
SNPs, where tables \ref{tab:snp_stats_per_spanner_single}
and \ref{tab:snp_stats_per_spanner_multi} show the
statistics for only single-spanners and multi-spanners respectively.
These probabilities are also displayed visually in 
figures \ref{fig:ppoisbinom} (all proteins), \ref{fig:ppoisbinom_single} (single-spanners)
and \ref{fig:ppoisbinom_multi} (multi-spanners).
 
% Label: tab:snp_stats
\begin{table}

\caption{\label{tab:snp_stats}Statistics for the multi-spanners. p = p value. n = number of SNPs. n\_success = number of SNPs found in TMHs (dashed blue line). E(n\_success) = expected number of SNPs to be found in TMHs (dashed red line). }
\centering
\begin{tabular}[t]{l|l}
\hline
parameter & value\\
\hline
p & 5.796491e-13\\
\hline
n & 21576\\
\hline
n\_success & 3831\\
\hline
E(n\_success) & 4229.928\\
\hline
\end{tabular}
\end{table}

% Label: tab:snp_stats_per_spanner_single
\begin{table}

\caption{\label{tab:snp_stats_per_spanner_single}Statistics for the single-spanners. p = p value. n = number of SNPs in single-spanners. n\_success = number of SNPs found in TMHs of single-spanners (dashed blue line). E(n\_success) = expected number of SNPs to be found in TMHs of single-spanners. }
\centering
\begin{tabular}[t]{l|l}
\hline
parameter & value\\
\hline
p & 0.3189532\\
\hline
n & 8186\\
\hline
n\_success & 452\\
\hline
E(n\_success) & 462.1535\\
\hline
\end{tabular}
\end{table}


% Label: tab:snp_stats_per_spanner_multi
\begin{table}

\caption{\label{tab:snp_stats_per_spanner_multi}Statistics for the multi-spanners. p = p value. n = number of SNPs in multi-spanners. n\_success = number of SNPs found in TMHs of multi-spanners (dashed blue line). E(n\_success) = expected number of SNPs to be found in TMHs of multi-spanners  (dashed line). }
\centering
\begin{tabular}[t]{l|l}
\hline
parameter & value\\
\hline
p & 8.315841e-12\\
\hline
n & 13022\\
\hline
n\_success & 3351\\
\hline
E(n\_success) & 3678.406\\
\hline
\end{tabular}
\end{table}


%
% Notes to self
%

%\begin{sidewaystable}
%  \centering
%  ... centered table here
%\end{sidewaystable}

% \scalebox{0.7}{
%  ... scaled table here
% }

%{\tiny
%  ... something with a tiny font
%}

% Instead of non-TMH use 'soluble protein region'
% or 'cytosolic or extracellular protein region'

